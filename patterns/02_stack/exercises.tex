\section{\Exercises}

\subsection{\Exercise \#1}
\label{exercise_stack_1}

\RU{Если это скомпилировать в MSVC и запустить, появится три числа. Откуда они берутся? 
Откуда они берутся если скомпилировать в MSVC с оптимизациями (\Ox)?}
\EN{If we compile this piece of code in MSVC and run it, three numbers are printed.
Where do they come from?
Where do they come from if you compile it in MSVC with optimization (\Ox)?}
\RU{Почему в GCC ситуация совсем иная}\EN{Why is the situation completely different if we compile with GCC}?

\begin{lstlisting}
#include <stdio.h>

int main()
{
	printf ("%d, %d, %d\n");

	return 0;
};
\end{lstlisting}

\Answer{}: \myref{exercise_solutions_stack_1}.

\subsection{\Exercise \#2}
\label{exercise_stack_2}

\WhatThisCodeDoes\

\begin{lstlisting}[caption=\Optimizing MSVC 2010]
$SG3103	DB	'%d', 0aH, 00H

_main	PROC
	push	0
	call	DWORD PTR __imp___time64
	push	edx
	push	eax
	push	OFFSET $SG3103 ; '%d'
	call	DWORD PTR __imp__printf
	add	esp, 16
	xor	eax, eax
	ret	0
_main	ENDP
\end{lstlisting}

\begin{lstlisting}[caption=\OptimizingKeilVI (\ARMMode)]
main PROC
        PUSH     {r4,lr}
        MOV      r0,#0
        BL       time
        MOV      r1,r0
        ADR      r0,|L0.32|
        BL       __2printf
        MOV      r0,#0
        POP      {r4,pc}
        ENDP

|L0.32|
        DCB      "%d\n",0
\end{lstlisting}

\begin{lstlisting}[caption=\OptimizingKeilVI (\ThumbMode)]
main PROC
        PUSH     {r4,lr}
        MOVS     r0,#0
        BL       time
        MOVS     r1,r0
        ADR      r0,|L0.20|
        BL       __2printf
        MOVS     r0,#0
        POP      {r4,pc}
        ENDP

|L0.20|
        DCB      "%d\n",0
\end{lstlisting}

\begin{lstlisting}[caption=\Optimizing GCC 4.9 (ARM64)]
main:
	stp	x29, x30, [sp, -16]!
	mov	x0, 0
	add	x29, sp, 0
	bl	time
	mov	x1, x0
	ldp	x29, x30, [sp], 16
	adrp	x0, .LC0
	add	x0, x0, :lo12:.LC0
	b	printf
.LC0:
	.string	"%d\n"
\end{lstlisting}

\lstinputlisting[caption=\Optimizing GCC 4.4.5 (MIPS) (IDA)]{patterns/02_stack/ex_MIPS_O3_IDA.lst}

\Answer{}: \myref{exercise_solutions_stack_2}.
