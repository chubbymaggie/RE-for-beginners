\chapter{\FPUChapterName}
\label{sec:FPU}

\newcommand{\FNURLSTACK}{\footnote{\href{http://go.yurichev.com/17123}{wikipedia.org/wiki/Stack\_machine}}}
\newcommand{\FNURLFORTH}{\footnote{\href{http://go.yurichev.com/17124}{wikipedia.org/wiki/Forth\_(programming\_language)}}}
\newcommand{\FNURLIEEE}{\footnote{\href{http://go.yurichev.com/17125}{wikipedia.org/wiki/IEEE\_floating\_point}}}
\newcommand{\FNURLSP}{\footnote{\href{http://go.yurichev.com/17126}{wikipedia.org/wiki/Single-precision\_floating-point\_format}}}
\newcommand{\FNURLDP}{\footnote{\href{http://go.yurichev.com/17127}{wikipedia.org/wiki/Double-precision\_floating-point\_format}}}
\newcommand{\FNURLEP}{\footnote{\href{http://go.yurichev.com/17128}{wikipedia.org/wiki/Extended\_precision}}}

\RU{\ac{FPU}\EMDASH блок в процессоре работающий с числами с плавающей запятой.}
\EN{The \ac{FPU} is a device within the main \ac{CPU}, specially designed to deal with floating point numbers.}
\RU{Раньше он назывался \q{сопроцессором} и он стоит немного в стороне от \ac{CPU}.}
\EN{It was called \q{coprocessor} in the past and it stays somewhat aside of the main \ac{CPU}.}

\section{IEEE 754}

\RU{Число с плавающей точкой в формате IEEE 754 состоит из \IT{знака}, \IT{мантиссы}\footnote{\IT{significand} или \IT{fraction} 
в англоязычной литературе} и \IT{экспоненты}.}
\EN{A number in the IEEE 754 format consists of a \IT{sign}, a \IT{significand} (also called \IT{fraction}) and an \IT{exponent}.}

\section{x86}

\RU{Перед изучением \ac{FPU} в x86 полезно ознакомиться с тем как работают стековые машины\FNURLSTACK 
или ознакомиться с основами языка Forth\FNURLFORTH.}
\EN{It is worth looking into stack machines\FNURLSTACK or learning the basics of the Forth language\FNURLFORTH,
before studying the \ac{FPU} in x86.}

\index{Intel!80486}
\index{Intel!FPU}
\RU{Интересен факт, что в свое время (до 80486) сопроцессор был отдельным чипом на материнской плате, 
и вследствие его высокой цены, он не всегда присутствовал. Его можно было докупить и установить отдельно}%
\EN{It is interesting to know that in the past (before the 80486 CPU) the coprocessor was a separate chip 
and it was not always pre-installed on the motherboard. It was possible to buy it separately and install it}%
\footnote{\RU{Например, Джон Кармак использовал в своей игре Doom числа с фиксированной запятой 
(\href{http://go.yurichev.com/17357}{ru.wikipedia.org/wiki/Число\_с\_фиксированной\_запятой}), хранящиеся
в обычных 32-битных \ac{GPR} (16 бит на целую часть и 16 на дробную),
чтобы Doom работал на 32-битных компьютерах без FPU, т.е. 80386 и 80486 SX.}
\EN{For example, John Carmack used fixed-point arithmetic 
(\href{http://go.yurichev.com/17356}{wikipedia.org/wiki/Fixed-point\_arithmetic}) values in his Doom video game, stored in 
32-bit \ac{GPR} registers (16 bit for integral part and another 16 bit for fractional part), so Doom
could work on 32-bit computers without FPU, i.e., 80386 and 80486 SX.}}.
\RU{Начиная с 80486 DX в состав процессора всегда входит FPU.}
\EN{Starting with the 80486 DX CPU, the \ac{FPU} is integrated in the \ac{CPU}.}

\index{x86!\Instructions!FWAIT}
\RU{Этот факт может напоминать такой рудимент как наличие инструкции \TT{FWAIT}, 
которая заставляет
\ac{CPU} ожидать, пока \ac{FPU} закончит работу}\EN{The \TT{FWAIT} instruction reminds us of that fact---it
switches the \ac{CPU} to a waiting state, so it can wait until the \ac{FPU} is done with its work}.
\RU{Другой рудимент это тот факт, что опкоды \ac{FPU}-инструкций начинаются с т.н. \q{escape}-опкодов 
(\TT{D8..DF}) как опкоды, передающиеся в отдельный сопроцессор.}
\EN{Another rudiment is the fact that the \ac{FPU} instruction 
opcodes start with the so called \q{escape}-opcodes (\TT{D8..DF}), i.e., 
opcodes passed to a separate coprocessor.}

\index{IEEE 754}
\label{FPU_is_stack}
\RU{FPU имеет стек из восьми 80-битных регистров:}
\EN{The FPU has a stack capable to holding 8 80-bit registers, and each register can hold a number 
in the IEEE 754\FNURLIEEE format.}
\RU{\ST{0}..\ST{7}. Для краткости, IDA и \olly отображают \ST{0} как \TT{ST},
что в некоторых учебниках и документациях означает \q{Stack Top} (\q{вершина стека}).}
\RU{Каждый регистр может содержать число в формате IEEE 754\FNURLIEEE.}
\EN{They are \ST{0}..\ST{7}. For brevity, IDA and \olly show \ST{0} as \TT{ST}, 
which is represented in some textbooks and manuals as \q{Stack Top}.}

\section{ARM, MIPS, x86/x64 SIMD}

\RU{В ARM и MIPS FPU это не стек, а просто набор регистров.}
\EN{In ARM and MIPS the FPU is not a stack, but a set of registers.}
\RU{Такая же идеология применяется в расширениях SIMD в процессорах x86/x64.}
\EN{The same ideology is used in the SIMD extensions of x86/x64 CPUs.}

\section{\CCpp}

\index{float}
\index{double}
\RU{В стандартных \CCpp имеются два типа для работы с числами с плавающей запятой: 
\Tfloat (\IT{число одинарной точности}\FNURLSP, 32 бита)
\footnote{Формат представления чисел с плавающей точкой одинарной точности затрагивается в разделе 
\IT{\WorkingWithFloatAsWithStructSubSubSectionName}~(\myref{sec:floatasstruct}).}
и \Tdouble (\IT{число двойной точности}\FNURLDP, 64 бита).}
\EN{The standard \CCpp languages offer at least two floating number types, \Tfloat (\IT{single-precision}\FNURLSP, 32 bits)
\footnote{the single precision floating point number format is also addressed in 
the \IT{\WorkingWithFloatAsWithStructSubSubSectionName}~(\myref{sec:floatasstruct}) section}
and \Tdouble (\IT{double-precision}\FNURLDP, 64 bits).}

\index{long double}
\RU{GCC также поддерживает тип \IT{long double} (\IT{extended precision}\FNURLEP, 80 бит), но MSVC~--- нет.}
\EN{GCC also supports the \IT{long double} type (\IT{extended precision}\FNURLEP, 80 bit), which MSVC doesn't.}

\RU{Несмотря на то, что \Tfloat занимает столько же места, сколько и \Tint на 32-битной архитектуре, 
представление чисел, разумеется, совершенно другое.}
\EN{The \Tfloat type requires the same number of bits as the \Tint type in 32-bit environments, 
but the number representation is completely different.}

\section{\RU{Простой пример}\EN{Simple example}}

\RU{Рассмотрим простой пример}\EN{Let's consider this simple example}:

\lstinputlisting{patterns/12_FPU/1_simple/simple.c}

\subsection{x86}

% subsubsections
\subsubsection{MSVC}

\RU{Компилируем в}\EN{Compile it in} MSVC 2010:

\lstinputlisting[caption=MSVC 2010: \ttf{}]{patterns/12_FPU/1_simple/MSVC.asm.\LANG}

\RU{\FLD берет 8 байт из стека и загружает их в регистр \ST{0}, автоматически конвертируя во внутренний 
80-битный формат (\IT{extended precision}).}
\EN{\FLD takes 8 bytes from stack and loads the number into the \ST{0} register, automatically converting 
it into the internal 80-bit format (\IT{extended precision}).}

\index{x86!\Instructions!FDIV}
\RU{\FDIV делит содержимое регистра \ST{0} на число, лежащее по адресу \TT{\_\_real@40091eb851eb851f}~--- 
там закодировано значение 3,14. Синтаксис ассемблера не поддерживает подобные числа, 
поэтому мы там видим шестнадцатеричное представление числа 3,14 в формате IEEE 754.}
\EN{\FDIV divides the value in \ST{0} by the number stored at address 
\TT{\_\_real@40091eb851eb851f}~---the value 3.14 is encoded there. 
The assembly syntax doesn't support floating point numbers, so 
what we see here is the hexadecimal representation of 3.14 in 64-bit IEEE 754 format.}

\RU{После выполнения \FDIV в \ST{0} остается \glslink{quotient}{частное}.}
\EN{After the execution of \FDIV \ST{0} holds the \gls{quotient}.}

\index{x86!\Instructions!FDIVP}
\RU{Кстати, есть ещё инструкция \FDIVP, которая делит \ST{1} на \ST{0}, 
выталкивает эти числа из стека и заталкивает результат. 
Если вы знаете язык Forth\FNURLFORTH, то это как раз оно и есть~--- стековая машина\FNURLSTACK.}
\EN{By the way, there is also the \FDIVP instruction, which divides \ST{1} by \ST{0}, 
popping both these values from stack and then pushing the result. 
If you know the Forth language\FNURLFORTH,
you can quickly understand that this is a stack machine\FNURLSTACK.}

\RU{Следующая \FLD заталкивает в стек значение $b$.}
\EN{The subsequent \FLD instruction pushes the value of $b$ into the stack.}

\RU{После этого в \ST{1} перемещается результат деления, а в \ST{0} теперь $b$.}
\EN{After that, the quotient is placed in \ST{1}, and \ST{0} has the value of $b$.}

\index{x86!\Instructions!FMUL}
\RU{Следующий \FMUL умножает $b$ из \ST{0} на значение \TT{\_\_real@4010666666666666}~--- 
там лежит число 4,1~--- и оставляет результат в \ST{0}.}
\EN{The next \FMUL instruction does multiplication: $b$ from \ST{0} is multiplied by by value at 
\TT{\_\_real@4010666666666666} (the numer 4.1 is there) and leaves the result in the \ST{0} register.}

\index{x86!\Instructions!FADDP}
\RU{Самая последняя инструкция \FADDP складывает два значения из вершины стека 
в \ST{1} и затем выталкивает значение, лежащее в \ST{0}. 
Таким образом результат сложения остается на вершине стека в \ST{0}.}
\EN{The last \FADDP instruction adds the two values at top of stack, storing the result in \ST{1} 
and then popping the value of \ST{0}, thereby leaving the result at the top of the stack, in \ST{0}.}

\RU{Функция должна вернуть результат в \ST{0}, так что больше ничего здесь не производится, 
кроме эпилога функции.}
\EN{The function must return its result in the \ST{0} register, 
so there are no any other instructions except the function epilogue after \FADDP.}

\ifdefined\IncludeOlly
\input{patterns/12_FPU/1_simple/olly.tex}
\fi

\ifdefined\IncludeGCC
\subsubsection{GCC}

\RU{GCC 4.4.1 (с опцией \Othree) генерирует похожий код, хотя и с некоторой разницей:}
\EN{GCC 4.4.1 (with \Othree option) emits the same code, just slightly different:}

\lstinputlisting[caption=\Optimizing GCC 4.4.1]{patterns/12_FPU/1_simple/GCC.asm.\LANG}

\RU{Разница в том, что в стек сначала заталкивается 3,14 (в \ST{0}), а затем значение 
из \TT{arg\_0} делится на то, что лежит в регистре \ST{0}.}
\EN{The difference is that, first of all, 3.14 is pushed to the stack (into \ST{0}), and then the value 
in \TT{arg\_0} is divided by the value in \ST{0}.}

\index{x86!\Instructions!FDIVR}
\RU{\FDIVR означает \IT{Reverse Divide}~--- делить, поменяв делитель и делимое местами. 
Точно такой же инструкции для умножения нет, потому что она была бы бессмысленна (ведь умножение 
операция коммутативная), так что остается только \FMUL без соответствующей ей \TT{-R} инструкции.}
\EN{\FDIVR stands for \IT{Reverse Divide}~---to divide with divisor and dividend swapped with each other. 
There is no likewise instruction for multiplication since it is 
a commutative operation, so we just have \FMUL without its \TT{-R} counterpart.}

\index{x86!\Instructions!FADDP}
\RU{\FADDP не только складывает два значения, но также и выталкивает из стека одно значение. 
После этого в \ST{0} остается только результат сложения.}
\EN{\FADDP adds the two values but also pops one value from the stack. 
After that operation, \ST{0} holds the sum.}
\fi


\ifdefined\IncludeARM
\subsection{ARM: \OptimizingXcodeIV (\ARMMode)}

\RU{Пока в ARM не было стандартного набора инструкций для работы с числами с плавающей точкой}%
\EN{Until ARM got standardized floating point support}, \RU{разные производители процессоров
могли добавлять свои расширения для работы с ними}\EN{several processor manufacturers added their own 
instructions extensions}.
\RU{Позже был принят стандарт}\EN{Then, } VFP (\IT{Vector Floating Point})\EN{ was standardized}.

\RU{Важное отличие от x86 в том, что там вы работаете с FPU-стеком, а здесь стека нет, 
вы работаете просто с регистрами.}
\EN{One important difference from x86 is that in ARM, there
is no stack, you work just with registers.}

\lstinputlisting[label=ARM_leaf_example10,caption=\OptimizingXcodeIV (\ARMMode)]{patterns/12_FPU/1_simple/ARM/Xcode_ARM_O3.asm.\LANG}

\index{ARM!D-\registers{}}
\index{ARM!S-\registers{}}
\RU{Итак, здесь мы видим использование новых регистров с префиксом D.}
\EN{So, we see here new some registers used, with D prefix.}
\RU{Это 64-битные регистры. Их 32 и их можно
использовать для чисел с плавающей точкой двойной точности (double) и для 
SIMD (в ARM это называется NEON).}
\EN{These are 64-bit registers, there are 32 of them, and they can be used both for floating-point numbers 
(double) but also for SIMD (it is called NEON here in ARM).}
\RU{Имеются также 32 32-битных S-регистра. Они применяются для работы с числами 
с плавающей точкой одинарной точности (float).}
\EN{There are also 32 32-bit S-registers, intended to be used for single precision 
floating pointer numbers (float).}
\RU{Запомнить легко: D-регистры предназначены для чисел double-точности, 
а S-регистры~--- для чисел single-точности.}
\EN{It is easy to remember: D-registers are for double precision numbers, while
S-registers---for single precision numbers.}
\RU{Больше об этом}\EN{More about it}: \myref{ARM_VFP_registers}.

\RU{Обе константы (3,14 и 4,1)}\EN{Both constants (3.14 and 4.1)} \RU{хранятся в памяти в формате IEEE 754.}
\EN{are stored in memory in IEEE 754 format.}

\index{ARM!\Instructions!VLDR}
\index{ARM!\Instructions!VMOV}
\RU{Инструкции }\TT{VLDR} \AndENRU \TT{VMOV}%
\RU{, как можно догадаться, это аналоги обычных \TT{LDR} и \MOV, но они работают с D-регистрами.}
\EN{, as it can be easily deduced, are analogous to the \TT{LDR} and \MOV instructions,
but they work with D-registers.}
\RU{Важно отметить, что эти инструкции, как и D-регистры, предназначены не только для работы 
с числами с плавающей точкой, но пригодны также и для работы с SIMD (NEON), и позже это также будет видно.}
\EN{It has to be noted that these instructions, just like the D-registers, are intended not only for
floating point numbers, 
but can be also used for SIMD (NEON) operations and this will also be shown soon.}

\RU{Аргументы передаются в функцию обычным путем через R-регистры, однако 
каждое число, имеющее двойную точность, занимает 64 бита, так что для передачи каждого нужны два R-регистра.}
\EN{The arguments are passed to the function in a common way, via the R-registers, however
each number that has double precision has a size of 64 bits, so two R-registers are needed to pass each one.}

\TT{VMOV D17, R0, R1} \RU{в самом начале составляет два 32-битных значения из \Reg{0} и \Reg{1} 
в одно 64-битное и сохраняет в}
\EN{at the start, composes two 32-bit values from \Reg{0} and \Reg{1} into one 64-bit value
and saves it to} \TT{D17}.

\TT{VMOV R0, R1, D16} \RU{в конце это обратная процедура}\EN{is the inverse operation}: 
\RU{то что было в}\EN{what was in} \TT{D16} 
\RU{остается в двух регистрах}\EN{is split in two registers,} \Reg{0} \AndENRU \Reg{1},
\RU{потому что}\EN{because} \RU{число с двойной точностью,}\EN{a double-precision number} 
\RU{занимающее 64 бита}\EN{that needs 64 bits for storage}, \RU{возвращается в паре регистров \Reg{0} и \Reg{1}.}
\EN{is returned in \Reg{0} and \Reg{1}.}

\index{ARM!\Instructions!VDIV}
\index{ARM!\Instructions!VMUL}
\index{ARM!\Instructions!VADD}
\TT{VDIV}, \TT{VMUL} \AndENRU \TT{VADD}, \RU{это инструкции для работы с числами 
с плавающей точкой, вычисляющие, соответственно, \glslink{quotient}{частное}, \glslink{product}{произведение} и сумму.}
\EN{are instruction for processing floating point numbers that compute \gls{quotient}, 
\gls{product} and sum, respectively.}

\RU{Код для Thumb-2 такой же.}\EN{The code for Thumb-2 is same.}

\subsection{ARM: \OptimizingKeilVI (\ThumbMode)}

\lstinputlisting{patterns/12_FPU/1_simple/ARM/Keil_O3_thumb.asm.\LANG}

\RU{Keil компилировал для процессора, в котором может и не быть поддержки FPU или NEON.}
\EN{Keil generated code for a processor without FPU or NEON support.}
\RU{Так что числа с двойной точностью передаются в парах обычных R-регистров,}
\EN{The double-precision floating-point numbers are passed via generic R-registers,}
\RU{а вместо FPU-инструкций вызываются сервисные библиотечные функции}
\EN{and instead of FPU-instructions, service library functions are called (like}
\TT{\_\_aeabi\_dmul}, \TT{\_\_aeabi\_ddiv}, \TT{\_\_aeabi\_dadd}%
\RU{, эмулирующие умножение, деление и сложение чисел с плавающей точкой.}
\EN{) which emulate multiplication, division and addition for floating-point numbers.}
\RU{Конечно, это медленнее чем FPU-сопроцессор, но лучше, чем ничего.}
\EN{Of course, that is slower than FPU-coprocessor, but still better than nothing.}

\RU{Кстати, похожие библиотеки для эмуляции сопроцессорных инструкций были очень распространены в x86 
когда сопроцессор был редким и дорогим и присутствовал далеко не во всех компьютерах.}
\EN{By the way, similar FPU-emulating libraries were very popular in the x86 world when coprocessors were rare
and expensive, and were installed only on expensive computers.}

\index{ARM!soft float}
\index{ARM!armel}
\index{ARM!armhf}
\index{ARM!hard float}
\RU{Эмуляция FPU-сопроцессора в ARM называется \IT{soft float} или \IT{armel} (\IT{emulation}),
а использование FPU-инструкций сопроцессора~--- \IT{hard float} или \IT{armhf}.}
\EN{The FPU-coprocessor emulation is called \IT{soft float} or \IT{armel} (\IT{emulation}) in the ARM world, 
while using the coprocessor's FPU-instructions is called \IT{hard float} or \IT{armhf}.}

\iffalse
% TODO разобраться...
\index{Raspberry Pi}
\RU{Ядро Linux, например, для Raspberry Pi может поставляться в двух вариантах.}
\EN{For example, the Linux kernel for Raspberry Pi is compiled in two variants.}
\RU{В случае \IT{soft float}, аргументы будут передаваться через R-регистры, 
а в случае \IT{hard float}, через D-регистры.}
\EN{In the \IT{soft float} case, arguments are passed via R-registers, and in the \IT{hard float} 
case---via D-registers.}

\RU{И это то, что помешает использовать, например, armhf-библиотеки
из armel-кода или наоборот, поэтому, весь код в дистрибутиве Linux должен быть скомпилирован
в соответствии с выбранным соглашением о вызовах.}
\EN{And that is what stops you from using armhf-libraries from armel-code or vice versa,
and that is
why all the code in Linux distributions must be compiled according to a single convention.}
\fi

\subsection{ARM64: \Optimizing GCC (Linaro) 4.9}

\RU{Очень компактный код}\EN{Very compact code}:

\lstinputlisting[caption=\Optimizing GCC (Linaro) 4.9]{patterns/12_FPU/1_simple/ARM/ARM64_GCC_O3.s.\LANG}

\subsection{ARM64: \NonOptimizing GCC (Linaro) 4.9}

\lstinputlisting[caption=\NonOptimizing GCC (Linaro) 4.9]{patterns/12_FPU/1_simple/ARM/ARM64_GCC_O0.s.\LANG}

\NonOptimizing GCC \RU{более многословный}\EN{is more verbose}.
\RU{Здесь много ненужных перетасовок значений, включая явно избыточный код 
(последние две инструкции \TT{GMOV}).}
\EN{There is a lot of unnecessary value shuffling, including some clearly redundant code 
(the last two \TT{FMOV} instructions).}
\RU{Должно быть}\EN{Probably}, GCC 4.9 \RU{пока ещё не очень хорош для генерации кода под ARM64}\EN{is not 
yet good in generating ARM64 code}.
\RU{Интересно заметить что у ARM64 64-битные регистры и D-регистры так же 64-битные.}
\EN{What is worth noting is that ARM64 has 64-bit registers, and the D-registers are 64-bit ones as well.}
\RU{Так что компилятор может сохранять значения типа \Tdouble в \ac{GPR} вместо локального стека.}
\EN{So the compiler is free to save values of type \Tdouble in \ac{GPR}s instead of the local stack.}
\RU{Это было невозможно на 32-битных CPU}\EN{This isn't possible on 32-bit CPUs}.

\RU{И снова, как упражнение, вы можете попробовать соптимизировать эту функцию вручную, без добавления
новых инструкций вроде \TT{FMADD}.}
\EN{And again, as an exercise, you can try to optimize this function manually, without introducing
new instructions like \TT{FMADD}.}

\fi
\ifdefined\IncludeMIPS
\subsection{MIPS}

\RU{MIPS может поддерживать несколько сопроцессоров (вплоть до 4), нулевой из которых это специальный
управляющий сопроцессор, а первый~--- это FPU.}
\EN{MIPS can support several coprocessors (up to 4), 
the zeroth of which is a special control coprocessor,
and first coprocessor is the FPU.}

\RU{Как и в ARM, сопроцессор в MIPS это не стековая машина. Он имеет 32 32-битных регистра (\$F0-\$F31):}
\EN{As in ARM, the MIPS coprocessor is not a stack machine, it has 32 32-bit registers (\$F0-\$F31):}
\myref{MIPS_FPU_registers}.
\RU{Когда нужно работать с 64-битными значениями типа \Tdouble, используется пара 32-битных F-регистров.}
\EN{When one needs to work with 64-bit \Tdouble values, a pair of 32-bit F-registers is used.}

\lstinputlisting[caption=\Optimizing GCC 4.4.5 (IDA)]{patterns/12_FPU/1_simple/MIPS_O3_IDA.lst.\LANG}

\RU{Новые инструкции}\EN{The new instructions here are}:

\begin{itemize}

\index{MIPS!\Instructions!LWC1}
\item LWC1 \RU{загружает 32-битное слово в регистр первого сопроцессора (отсюда \q{1} в названии инструкции).}
\EN{loads a 32-bit word into a register of the first coprocessor (hence \q{1} in instruction name).}
\index{MIPS!\Pseudoinstructions!L.D}
\RU{Пара инструкций LWC1 может быть объединена в одну псевдоинструкцию L.D.}
\EN{A pair of LWC1 instructions may be combined into a L.D pseudoinstruction.}

\index{MIPS!\Instructions!DIV.D}
\index{MIPS!\Instructions!MUL.D}
\index{MIPS!\Instructions!ADD.D}
\item DIV.D, MUL.D, ADD.D \RU{производят деление, умножение и сложение соответственно}\EN{do division, multiplication, and addition respectively} 
(\q{.D} \RU{в суффиксе означает двойную точность}\EN{in the suffix stands for double precision}, 
\q{.S}\RU{~--- одинарную точность}\EN{ stands for single precision})

\end{itemize}

\index{MIPS!\Instructions!LUI}
\index{\CompilerAnomaly}
\label{MIPS_FPU_LUI}
\RU{Здесь также имеется странная аномалия компилятора: инструкция \INS{LUI} помеченная нами вопросительным знаком.}%
\EN{There is also a weird compiler anomaly: the \INS{LUI} instructions that we've marked with a question mark.}
\RU{Мне трудно понять, зачем загружать часть 64-битной константы типа \Tdouble в регистр \$V0.}%
\EN{It's hard for me to understand why load a part of a 64-bit constant of \Tdouble type into the \$V0 register.}
\RU{От этих инструкций нет толка}\EN{These instruction have no effect}.
% TODO did you try checking out compiler source code?
\RU{Если кто-то об этом что-то знает, пожалуйста, напишите автору емейл}%
\EN{If someone knows more about it, please drop an email to author}\footnote{\EMAIL}.

\fi

\section{\RU{Передача чисел с плавающей запятой в аргументах}\EN{Passing floating point numbers via arguments}}
\index{\CStandardLibrary!pow()}

\lstinputlisting{patterns/12_FPU/2_passing_floats/pow.c}

\subsection{x86}

\RU{Посмотрим, что у нас вышло}\EN{Let's see what we get in} (MSVC 2010):

\lstinputlisting[caption=MSVC 2010]{patterns/12_FPU/2_passing_floats/MSVC.asm.\LANG}

\index{x86!\Instructions!FLD}
\index{x86!\Instructions!FSTP}
\RU{\FLD и \FSTP перемещают переменные из сегмента данных в FPU-стек или обратно. 
\TT{pow()}\footnote{стандартная функция Си, возводящая число в степень} достает оба значения из FPU-стека и 
возвращает результат в \ST{0}. 
\printf берет 8 байт из стека и трактует их как переменную типа \Tdouble.}
\EN{\FLD and \FSTP move variables between the data segment and the FPU stack. 
\TT{pow()}\footnote{a standard C function, raises a number to the given power (exponentiation)}
takes both values from the stack of the FPU and 
returns its result in the \ST{0} register.
\printf takes 8 bytes from the local stack and interprets them as \Tdouble type variable.}

\ifdefined\IncludeARM
\RU{Кстати, с тем же успехом можно было бы перекладывать эти два числа из памяти в стек
при помощи пары \MOV:}
\EN{By the way, a pair of \MOV instructions could be used here for moving values from the memory
into the stack,} 
\RU{ведь в памяти числа в формате IEEE 754, pow() также принимает их в том же
формате, и никакая конверсия не требуется.}
\EN{because the values in memory are stored in IEEE 754 format, and pow() also takes them in this
format, so no conversion is necessary.}
\RU{Собственно, так и происходит в следующем примере с ARM}%
\EN{That's how it's done in the next example, for ARM}:
\myref{FPU_passing_floats_ARM}.
\fi

\ifdefined\IncludeARM
\subsection{ARM + \NonOptimizingXcodeIV (\ThumbTwoMode)}
\label{FPU_passing_floats_ARM}

\lstinputlisting{patterns/12_FPU/2_passing_floats/Xcode_thumb_O0.asm}

\RU{Как уже было указано, 64-битные числа с плавающей точкой передаются в парах R-регистров.}%
\EN{As it was mentioned before, 64-bit floating pointer numbers are passed in R-registers pairs.}
\RU{Этот код слегка избыточен (наверное, потому что не включена оптимизация), ведь можно было бы 
загружать значения напрямую в R-регистры минуя загрузку в D-регистры.}
\EN{This code is a bit redundant (certainly because optimization is turned off), 
since it is possible to load values into the R-registers directly without touching the D-registers.}

\RU{Итак, видно, что функция}\EN{So, as we see, the} \TT{\_pow} \RU{получает первый аргумент в}
\EN{function receives its first argument in} \Reg{0} \AndENRU \Reg{1}, \RU{а второй в}\EN{and its second one in} 
\Reg{2} \AndENRU \Reg{3}. 
\RU{Функция оставляет результат в}\EN{The function leaves its result in} \Reg{0} \AndENRU \Reg{1}.
\RU{Результат работы}\EN{The result of} \TT{\_pow} \RU{перекладывается в}\EN{is moved into} \TT{D16}, 
\RU{затем в пару}\EN{then in the} \Reg{1} \AndENRU \Reg{2}\EN{ pair}, \RU{откуда}\EN{from where} 
\printf \RU{берет это число-результат.}
\EN{takes the resulting number.}

\subsection{ARM + \NonOptimizingKeilVI (\ARMMode)}

\lstinputlisting{patterns/12_FPU/2_passing_floats/Keil_ARM_O0.asm}

\RU{Здесь не используются D-регистры, используются только пары R-регистров.}
\EN{D-registers are not used here, just R-register pairs.}

\subsection{ARM64 + \Optimizing GCC (Linaro) 4.9}

\lstinputlisting[caption=\Optimizing GCC (Linaro) 4.9]{patterns/12_FPU/2_passing_floats/ARM64.s.\LANG}

\RU{Константы загружаются в}\EN{The constants are loaded into} \RegD{0} \AndENRU \RegD{1}: 
\RU{функция }pow() \RU{берет их оттуда}\EN{takes them from there}.
\RU{Результат в}\EN{The result will be in} \RegD{0} \RU{после исполнения}\EN{after the execution of} pow().
\RU{Он пропускается в}\EN{It is to be passed to} \printf \RU{без всякой модификации и перемещений}\EN{without 
any modification and moving}, 
\RU{потому что}\EN{because} \printf \RU{берет аргументы \glslink{integral type}{интегральных типов} и указатели 
из X-регистров,
а аргументы типа плавающей точки из D-регистров}\EN{takes arguments of \glslink{integral type}{integral types} 
and pointers from X-registers, and floating point arguments from D-registers}.

\fi
\ifdefined\IncludeMIPS
\subsection{MIPS}

\lstinputlisting[caption=\Optimizing GCC 4.4.5 (IDA)]{patterns/12_FPU/2_passing_floats/MIPS_O3_IDA.lst.\LANG}

\RU{И снова мы здесь видим, как LUI загружает 32-битную часть числа типа \Tdouble в \$V0.}
\EN{And again, we see here LUI loading a 32-bit part of a \Tdouble number into \$V0.}
\RU{И снова трудно понять почему}\EN{And again, it's hard to comprehend why}.

\index{MIPS!\Instructions!MFC1}
\RU{Новая для нас инструкция это}
\EN{The new instruction for us here is} \INS{MFC1} (\q{Move From Coprocessor 1})\RU{ (копировать из первого сопроцессора)}.
\RU{FPU это сопроцессор под номером 1, вот откуда \q{1} в имени инструкции.}
\EN{The FPU is coprocessor number 1, hence \q{1} in the instruction name.}
\RU{Эта инструкция переносит значения из регистров сопроцессора в регистры основного CPU (\ac{GPR}).}
\EN{This instruction transfers values from the coprocessor's registers to the registers of the CPU (\ac{GPR}).}
\RU{Так что результат исполнения pow() в итоге копируется в регистры \$A3 и \$A2
и из этой пары регистров \printf берет его как 64-битное значение типа \Tdouble.}
\EN{So in the end the result from pow() is moved to registers \$A3 and \$A2, 
and \printf takes a 64-bit double value from this register pair.}

\fi

\section{\RU{Пример с сравнением}\EN{Comparison example}}

\RU{Попробуем теперь вот это:}\EN{Let's try this:}

\lstinputlisting{patterns/12_FPU/3_comparison/d_max.c}

\RU{Несмотря на кажущуюся простоту этой функции, понять, как она работает, будет чуть сложнее.}
\EN{Despite the simplicity of the function, it will be harder to understand how it works.}

% subsections
\subsection{x86}

% subsubsections
\subsubsection{\NonOptimizing MSVC}

\RU{Вот что выдал MSVC 2010}\EN{MSVC 2010 generates the following}:

\lstinputlisting[caption=\NonOptimizing MSVC 2010]{patterns/12_FPU/3_comparison/x86/MSVC/MSVC.asm.\LANG}

\index{x86!\Instructions!FLD}
\RU{Итак, \FLD загружает \TT{\_b} в регистр \ST{0}.}
\EN{So, \FLD loads \TT{\_b} into \ST{0}.}

\label{Czero_etc}
\newcommand{\Czero}{\TT{C0}\xspace}
\newcommand{\Ctwo}{\TT{C2}\xspace}
\newcommand{\Cthree}{\TT{C3}\xspace}
\newcommand{\CThreeBits}{\Cthree/\Ctwo/\Czero}

\index{x86!\Instructions!FCOMP}
\RU{\FCOMP сравнивает содержимое \ST{0} с тем, что лежит в \TT{\_a} и выставляет биты \CThreeBits в 
регистре статуса FPU. Это 16-битный регистр отражающий текущее состояние FPU.}
\EN{\FCOMP compares the value in \ST{0} with what is in \TT{\_a} 
and sets \CThreeBits bits in FPU 
status word register, accordingly. 
This is a 16-bit register that reflects the current state of the FPU.}

\RU{После этого инструкция \FCOMP также выдергивает одно значение из стека. 
Это отличает её от \FCOM, которая просто сравнивает значения, оставляя стек в таком же состоянии.}
\EN{After the bits are set, the \FCOMP instruction also pops one variable from the stack. 
This is what distinguishes it from \FCOM, which is just compares values, leaving the stack in the same state.}

\RU{К сожалению, у процессоров до Intel P6%
\footnote{Intel P6 это Pentium Pro, Pentium II, и последующие модели} нет инструкций условного перехода,
проверяющих биты \CThreeBits.
Возможно, так сложилось исторически (вспомните о том, что FPU когда-то был вообще отдельным чипом).\\
А у Intel P6 появились инструкции \FCOMI/\FCOMIP/\FUCOMI/\FUCOMIP, делающие то же самое, 
только напрямую модифицирующие флаги \ZF/\PF/\CF.}
\EN{Unfortunately, CPUs before Intel P6
\footnote{Intel P6 is Pentium Pro, Pentium II, etc} don't have any conditional 
jumps instructions which check the \CThreeBits bits. 
Probably, it is a matter of history (remember: FPU was separate chip in past).\\
Modern CPU starting at Intel P6 have \FCOMI/\FCOMIP/\FUCOMI/\FUCOMIP 
instructions~---which do the same, but modify the \ZF/\PF/\CF CPU flags.}

\index{x86!\Instructions!FNSTSW}
\RU{Так что \FNSTSW копирует содержимое регистра статуса в \AX. 
Биты \CThreeBits занимают позиции, 
соответственно, 14, 10, 8. В этих позициях они и остаются в регистре \AX, 
и все они расположены в старшей части регистра~--- \AH.}
\EN{The \FNSTSW instruction copies FPU the status word register to \AX. 
\CThreeBits bits are placed at positions 14/10/8, 
they are at the same positions in the \AX register and all they are placed in the high part of \AX{}~---\AH{}.}

\begin{itemize}
\item
\RU{Если $b>a$ в нашем случае, то биты \CThreeBits должны быть выставлены так:}
\EN{If $b>a$ in our example, then \CThreeBits bits are to be set as following:} 0, 0, 0.
\item
\RU{Если $a>b$, то биты будут выставлены:}\EN{If $a>b$, then the bits are:} 0, 0, 1.
\item
\RU{Если $a=b$, то биты будут выставлены так:}\EN{If $a=b$, then the bits are:} 1, 0, 0.
\item
\RU{Если результат не определен (в случае ошибки), то биты будут выставлены так:}
\EN{If the result is unordered (in case of error), then the set bits are:} 1, 1, 1.
\end{itemize}
% TODO: table here?

\EN{This is how \CThreeBits bits are located in the \AX register:}
\RU{Вот как биты \CThreeBits расположены в регистре \AX:}

\input{C3_in_AX}

\EN{This is how \CThreeBits bits are located in the \AH register:}
\RU{Вот как биты \CThreeBits расположены в регистре \AH:}

\input{C3_in_AH}

\RU{После исполнения \TT{test ah, 5}\footnote{5=101b} % FIXME: subscript here!
будут учтены только биты \Czero и \Ctwo (на позициях 0 и 2), остальные просто проигнорированы.}
\EN{After the execution of \TT{test ah, 5}\footnote{5=101b}, 
only \Czero and \Ctwo bits (on 0 and 2 position) are considered, all other bits are just
ignored.}

\label{parity_flag}
\index{x86!\Registers!\Flags!\RU{Флаг четности}\EN{Parity flag}}
\RU{Теперь немного о \IT{parity flag}\footnote{флаг четности}. 
Ещё один замечательный рудимент эпохи.}
\EN{Now let's talk about the \IT{parity flag}, another notable historical rudiment.}

\RU{Этот флаг выставляется в 1 если количество единиц в последнем результате четно. 
И в 0 если нечетно.}
\EN{This flag is set to 1 if the number of ones in the result of the last calculation is even, 
and to 0 if it is odd.}

\RU{Заглянем в}\EN{Let's look into} Wikipedia%
\footnote{\href{http://go.yurichev.com/17131}{wikipedia.org/wiki/Parity\_flag}}:

\begin{framed}
\begin{quotation}
One common reason to test the parity flag actually has nothing to do with parity. The FPU has four condition flags 
(C0 to C3), but they can not be tested directly, and must instead be first copied to the flags register. 
When this happens, C0 is placed in the carry flag, C2 in the parity flag and C3 in the zero flag. 
The C2 flag is set when e.g. incomparable floating point values (NaN or unsupported format) are compared 
with the FUCOM instructions.
\end{quotation}
\end{framed}

\EN{As noted in Wikipedia, the parity flag used sometimes in FPU code, let's see how.}
\RU{Как упоминается в Wikipedia, флаг четности иногда используется в FPU-коде и сейчас мы увидим как.}

\index{x86!\Instructions!JP}
\RU{Флаг \PF будет выставлен в 1, если \Czero и \Ctwo 
оба 1 или оба 0. 
И тогда сработает последующий \JP (\IT{jump if PF==1}). 
Если мы вернемся чуть назад и посмотрим значения \CThreeBits 
для разных вариантов, то увидим, что условный переход \JP сработает в двух случаях: если $b>a$ или если $a=b$ 
(ведь бит \Cthree перестал учитываться после исполнения \TT{test ah, 5}).}
\EN{The \PF flag is to be set to 1 if both \Czero and \Ctwo are set to 0 or both are 1, in which case
the subsequent \JP (\IT{jump if PF==1}) is triggering. 
If we recall the values of \CThreeBits for various cases,
we can see that the conditional jump 
\JP is triggering in two cases: if $b>a$ or $a=b$ 
(\Cthree bit is not considered here, since it was cleared by 
the \TT{test ah, 5} instruction).}

\RU{Дальше всё просто. Если условный переход сработал, то \FLD загрузит значение \TT{\_b} в \ST{0}, 
а если не сработал, то загрузится \TT{\_a} и произойдет выход из функции.}
\EN{It is all simple after that. 
If the conditional jump was triggered, 
\FLD loads the value of \TT{\_b} 
in \ST{0}, and if it was not triggered, the value of \TT{\_a} is loaded there.}

\myparagraph{\RU{А как же проверка флага \Ctwo}\EN{And what about checking \Ctwo}?}

\RU{Флаг \Ctwo включается в случае ошибки (\gls{NaN}, \etc{}.), но наш код его не проверяет.}
\EN{The \Ctwo flag is set in case of error (\gls{NaN}, \etc{}), but our code doesn't check it.}
\RU{Если программисту нужно знать, не произошла ли FPU-ошибка, он должен позаботиться об этом
дополнительно, добавив соответствующие проверки.}
\EN{If the programmer cares about FPU errors, he/she must add additional checks.}

\ifdefined\IncludeOlly
\input{patterns/12_FPU/3_comparison/x86/MSVC/olly.tex}
\fi

\subsubsection{\Optimizing MSVC 2010}

\lstinputlisting[caption=\Optimizing MSVC 2010]{patterns/12_FPU/3_comparison/x86/MSVC_Ox/MSVC.asm.\LANG}

\index{x86!\Instructions!FCOM}
\RU{\FCOM отличается от \FCOMP тем, что просто сравнивает значения и оставляет стек в том же состоянии. 
В отличие от предыдущего примера, операнды здесь в обратном порядке. 
Поэтому и результат сравнения в \CThreeBits будет отличаться:}
\EN{\FCOM differs from \FCOMP in the sense that it just compares the values and doesn't change the FPU stack. 
Unlike the previous example, here the operands are in reverse order, 
which is why the result of the comparison in \CThreeBits is different:}

\begin{itemize}
\item
\RU{Если $a>b$, то биты \CThreeBits должны быть выставлены так:}
\EN{If $a>b$ in our example, then \CThreeBits bits are to be set as:} 0, 0, 0.
\item
\RU{Если $b>a$, то биты будут выставлены так:}\EN{If $b>a$, then the bits are:} 0, 0, 1.
\item
\RU{Если $a=b$, то биты будут выставлены так:}\EN{If $a=b$, then the bits are:} 1, 0, 0.
\end{itemize}
% TODO: table?

\RU{Инструкция \TT{test ah, 65} как бы оставляет только два бита~--- \Cthree и \Czero. 
Они оба будут нулями, если $a>b$: в таком случае переход \JNE не сработает. 
\index{ARM!\Instructions!FSTP}
Далее имеется инструкция \TT{FSTP ST(1)}~--- эта инструкция копирует 
значение \ST{0} в указанный операнд и выдергивает одно значение из стека. В данном случае, 
она копирует \ST{0} 
(где сейчас лежит~\TT{\_a})~в~\ST{1}. 
После этого на вершине стека два раза лежит~\TT{\_a}. Затем одно значение выдергивается. 
После этого в \ST{0} остается~\TT{\_a} и функция завершается.}
\EN{The \TT{test ah, 65} instruction leaves just two bits~---\Cthree and \Czero. 
Both will be zero if $a>b$: in that case the \JNE jump will not be triggered. 
Then \TT{FSTP ST(1)} follows~---this instruction copies the value from \ST{0} to the operand and 
pops one value from the FPU stack.
In other words, the instruction copies \ST{0} (where the value of \TT{\_a} is now) into \ST{1}.
After that, two copies of {\_a} are at the top of the stack. 
Then, one value is popped.
After that, \ST{0} contains {\_a} and the function is finishes.}

\RU{Условный переход \JNE сработает в двух других случаях: если $b>a$ или $a=b$. 
\ST{0} скопируется в \ST{0} (как бы холостая операция). 
Затем одно значение из стека вылетит и на вершине стека останется то, что 
до этого лежало в \ST{1} (то~есть~\TT{\_b}). И функция завершится. 
Эта инструкция используется здесь видимо потому что в FPU 
нет другой инструкции, которая просто выдергивает 
значение из стека и выбрасывает его.}
\EN{The conditional jump \JNE is triggering in two cases: if $b>a$ or $a=b$. 
\ST{0} is copied into \ST{0}, it is just like an idle (\ac{NOP}) operation, then one value 
is popped from the stack and the top of the stack (\ST{0}) is contain what was in \ST{1} before 
(that is {\_b}). 
Then the function finishes. 
The reason this instruction is used here probably is because the \ac{FPU} 
has no other instruction to pop a value from the stack and discard it.}

\ifdefined\IncludeOlly
\input{patterns/12_FPU/3_comparison/x86/MSVC_Ox/olly.tex}
\fi

\ifdefined\IncludeGCC
\subsubsection{GCC 4.4.1}

\lstinputlisting[caption=GCC 4.4.1]{patterns/12_FPU/3_comparison/x86/GCC.asm.\LANG}

\index{x86!\Instructions!FUCOMPP}
\RU{\FUCOMPP~--- это почти то же что и \FCOM, только выкидывает из стека оба значения после сравнения, 
а также несколько иначе реагирует на \q{не-числа}.}
\EN{\FUCOMPP{} is almost like \FCOM, but pops both values from the stack and handles
\q{not-a-numbers} differently.}

\index{\RU{Не-числа}\EN{Non-a-numbers} (NaNs)}
\RU{Немного о \IT{не-числах}}\EN{A bit about \IT{not-a-numbers}}.

\newcommand{\NANFN}{\footnote{\RU{\href{http://go.yurichev.com/17129}{ru.wikipedia.org/wiki/NaN}}%
\EN{\href{http://go.yurichev.com/17130}{wikipedia.org/wiki/NaN}}}}

\RU{FPU умеет работать со специальными переменными, которые числами не являются и называются \q{не числа} или 
\gls{NaN}\NANFN. 
Это бесконечность, результат деления на ноль, и так далее. Нечисла бывают \q{тихие} и \q{сигнализирующие}. 
С первыми можно продолжать работать и далее, а вот если вы попытаетесь совершить какую-то операцию 
с сигнализирующим нечислом, то сработает исключение.}
\EN{The FPU is able to deal with special values which are \IT{not-a-numbers} or 
\gls{NaN}s\NANFN. 
These are infinity, result of division by 0, etc. 
Not-a-numbers can be \q{quiet} and \q{signaling}. It is possible to continue to work with \q{quiet} NaNs, 
but if one tries to do any operation with \q{signaling} NaNs, an exception is to be raised.}

\index{x86!\Instructions!FCOM}
\index{x86!\Instructions!FUCOM}
\RU{Так вот, \FCOM вызовет исключение если любой из операндов какое-либо нечисло.
\FUCOM же вызовет исключение только если один из операндов именно \q{сигнализирующее нечисло}.}
\EN{\FCOM raising an exception if any operand is \gls{NaN}. 
\FUCOM raising an exception only if any operand is a signaling \gls{NaN} (SNaN).}

\index{x86!\Instructions!SAHF}
\label{SAHF}
\RU{Далее мы видим \SAHF (\IT{Store AH into Flags})~--- это довольно редкая инструкция в коде, не использующим FPU. 
8 бит из \AH перекладываются в младшие 8 бит регистра статуса процессора в таком порядке:}
\EN{The next instruction is \SAHF (\IT{Store AH into Flags})~---this is a rare 
instruction in code not related to the FPU. 
8 bits from AH are moved into the lower 8 bits of the CPU flags in the following order:}

\input{SAHF_LAHF}

\index{x86!\Instructions!FNSTSW}
\RU{Вспомним, что \FNSTSW перегружает интересующие нас биты \CThreeBits в \AH, 
и соответственно они будут в позициях 6, 2, 0 в регистре \AH:}
\EN{Let's recall that \FNSTSW moves the bits that interest us (\CThreeBits) into \AH 
and they are in positions 6, 2, 0 of the \AH register:}

\input{C3_in_AH}

\RU{Иными словами, пара инструкций \TT{fnstsw  ax / sahf} перекладывает биты \CThreeBits в флаги \ZF, \PF, \CF.}
\EN{In other words, the \TT{fnstsw  ax / sahf} instruction pair moves \CThreeBits into \ZF, \PF and \CF.}

\RU{Теперь снова вспомним, какие значения бит \CThreeBits будут при каких результатах сравнения:}
\EN{Now let's also recall the values of \CThreeBits in different conditions:}

\begin{itemize}
\item
\RU{Если $a$ больше $b$ в нашем случае, то биты \CThreeBits должны быть выставлены так:}
\EN{If $a$ is greater than $b$ in our example, then \CThreeBits are to be set to:} 0, 0, 0.
\item
\RU{Если $a$ меньше $b$, то биты будут выставлены так:}
\EN{if $a$ is less than $b$, then the bits are to be set to:} 0, 0, 1.
\item
\RU{Если $a=b$, то так:}\EN{If $a=b$, then:} 1, 0, 0.
\end{itemize}
% TODO: table?

\RU{Иными словами, после трех инструкций \FUCOMPP/\FNSTSW/\SAHF 
возможны такие состояния флагов:}
\EN{In other words, these states of the CPU flags are possible
after three \FUCOMPP/\FNSTSW/\SAHF instructions:}

\begin{itemize}
\item
\RU{Если $a>b$ в нашем случае, то флаги будут выставлены так:}
\EN{If $a>b$, the CPU flags are to be set as:} \TT{ZF=0, PF=0, CF=0}.
\item
\RU{Если $a<b$, то флаги будут выставлены так:}
\EN{If $a<b$, then the flags are to be set as:} \TT{ZF=0, PF=0, CF=1}.
\item
\RU{Если $a=b$, то так:}\EN{And if $a=b$, then:} \TT{ZF=1, PF=0, CF=0}.
\end{itemize}
% TODO: table?

\index{x86!\Instructions!SETcc}
\index{x86!\Instructions!JNBE}
\RU{Инструкция \SETNBE выставит в \AL единицу или ноль в зависимости от флагов и условий. 
Это почти аналог \JNBE, за тем лишь исключением, что \SETcc
\footnote{\IT{cc} это \IT{condition code}}
выставляет 1 или 0 в \AL, а \Jcc делает переход или нет. 
\SETNBE запишет 1 только если \TT{CF=0} и \TT{ZF=0}. Если это не так, то запишет 0 в \AL.}
\EN{Depending on the CPU flags and conditions, \SETNBE stores 1 or 0 to AL. 
It is almost the counterpart of \JNBE, with the exception that \SETcc 
\footnote{\IT{cc} is \IT{condition code}} stores 1 or 0 in \AL, 
but \Jcc does actually jump or not. 
\SETNBE stores 1 only if \TT{CF=0} and \TT{ZF=0}. 
If it is not true, 0 is to be stored into \AL.}

\RU{\CF будет 0 и \ZF будет 0 одновременно только в одном случае: если $a>b$.}
\EN{Only in one case both \CF and \ZF are 0: if $a>b$.}

\RU{Тогда в \AL будет записана 1, последующий условный переход \JZ выполнен не будет 
и функция вернет~\TT{\_a}. 
В остальных случаях, функция вернет~\TT{\_b}.}
\EN{Then 1 is to be stored to \AL, 
the subsequent \JZ is not to be triggered and the function will return {\_a}. 
In all other cases, {\_b} is to be returned.}
\fi

\subsubsection{\Optimizing GCC 4.4.1}

\lstinputlisting[caption=\Optimizing GCC 4.4.1]{patterns/12_FPU/3_comparison/x86/GCC_O3.asm.\LANG}

\index{x86!\Instructions!JA}
\RU{Почти всё что здесь есть, уже описано мною, кроме одного: использование \JA после \SAHF. 
Действительно, инструкции условных переходов \q{больше}, \q{меньше} и \q{равно} для сравнения беззнаковых чисел 
(а это \JA, \JAE, \JB, \JBE, \JE/\JZ, \JNA, \JNAE, \JNB, \JNBE, \JNE/\JNZ) проверяют только флаги \CF и \ZF.}
\EN{It is almost the same except that \JA is used after \SAHF. 
Actually, conditional jump instructions that check \q{larger}, \q{lesser} or \q{equal} for unsigned number comparison 
(these are \JA, \JAE, \JB, \JBE, \JE/\JZ, \JNA, \JNAE, \JNB, \JNBE, \JNE/\JNZ) check only flags \CF and \ZF.}\ESph{}\PTBRph{}\PLph{}\\
\\
\EN{Let's recall where bits \CThreeBits are located in the \TT{AH} register after the execution of \TT{FSTSW}/\FNSTSW:}
\RU{Вспомним, как биты \CThreeBits располагаются в регистре \TT{AH} после исполнения \TT{FSTSW}/\FNSTSW:}

\input{C3_in_AH}

\RU{Вспомним также, как располагаются биты из \TT{AH} во флагах CPU после исполнения \SAHF:}
\EN{Let's also recall, how the bits from \TT{AH} are stored into the CPU flags the execution of \SAHF:}

\input{SAHF_LAHF}

\RU{Биты \Cthree и \Czero после сравнения перекладываются в флаги \ZF и \CF так, 
что перечисленные инструкции переходов могут работать. 
\JA сработает, если \CF и \ZF обнулены.}
\EN{After the comparison, the \Cthree and \Czero bits are moved into \ZF and \CF,
so the conditional jumps are able work after.
\JA is triggering if both \CF are \ZF zero.}

\RU{Таким образом, перечисленные инструкции условного перехода можно использовать после инструкций \FNSTSW/\SAHF.}
\EN{Thereby, the conditional jumps instructions listed here can be used after a \FNSTSW/\SAHF instruction pair.}

\RU{Может быть, биты статуса FPU \CThreeBits преднамеренно были размещены таким образом, 
чтобы переноситься на базовые флаги процессора без перестановок?}
\EN{Apparently, the FPU \CThreeBits status bits were placed there intentionally, 
to easily map them to base CPU flags without additional permutations?}

\ifdefined\IncludeGCC
\subsubsection{GCC 4.8.1 \RU{с оптимизацией \Othree}\EN{with \Othree optimization turned on}}
\label{gcc481_o3}

\EN{Some new FPU instructions were added in the P6 Intel family}\RU{В линейке процессоров P6 от Intel 
появились новые FPU-инструкции}\footnote{\EN{Starting at}\RU{Начиная с} Pentium Pro, 
Pentium-II, \etc.}.
\index{x86!\Instructions!FUCOMI}
\RU{Это}\EN{These are} \TT{FUCOMI} (\RU{сравнить операнды и выставить флаги основного CPU}\EN{compare 
operands and set flags of the main CPU}) \AndENRU 
\index{x86!\Instructions!FCMOVcc}
\TT{FCMOVcc} (\RU{работает как}\EN{works like} \TT{CMOVcc}, \RU{но на регистрах FPU}\EN{but on FPU registers}).
\RU{Очевидно, разработчики GCC решили отказаться от поддержки процессоров до линейки P6 (ранние Pentium, 80486, etc{}.)}
\EN{Apparently, the maintainers of GCC decided to drop support of pre-P6 Intel CPUs (early Pentiums, 80486, etc{})}.

\RU{И кстати, FPU уже давно не отдельная часть процессора в линейке P6, так что флаги основного CPU можно
модифицировать из FPU.}
\EN{And also, the FPU is no longer separate unit in P6 Intel family, so now it is possible to modify/check flags 
of the main CPU from the FPU.}

\RU{Вот что имеем}\EN{So what we get is}:

\lstinputlisting[caption=\Optimizing GCC 4.8.1]{patterns/12_FPU/3_comparison/x86/GCC481_O3.s.\LANG}

\RU{Не совсем понимаю, зачем здесь \TT{FXCH} (поменять местами операнды).}
\EN{Hard to guess why \TT{FXCH} (swap operands) is here.}
\RU{От нее легко избавиться поменяв местами инструкции \FLD либо заменив 
\TT{FCMOVBE} (\IT{below or equal}~--- меньше или равно) на 
\TT{FCMOVA} (\IT{above}~--- больше).}
\EN{It's possible to get rid of it easily by swapping the first two \FLD instructions or by replacing 
\TT{FCMOVBE} (\IT{below or equal}) by \TT{FCMOVA} (\IT{above}).}
\RU{Должно быть, неаккуратность компилятора}\EN{Probably it's a compiler inaccuracy}.

\RU{Так что}\EN{So} \TT{FUCOMI} \EN{compares}\RU{сравнивает} \ST{0} ($a$) \AndENRU \ST{1} ($b$) 
\RU{и затем устанавливает флаги основного CPU}\EN{and then sets some flags in the main CPU}.
\TT{FCMOVBE} \RU{проверяет флаги и копирует}\EN{checks the flags and copies} \ST{1} 
(\RU{в тот момент там находится $b$}\EN{$b$ here at the moment}) \RU{в}\EN{to} 
\ST{0} (\RU{там $a$}\EN{$a$ here}) \RU{если}\EN{if} $ST0 (a) <= ST1 (b)$.
\RU{В противном случае}\EN{Otherwise} ($a>b$), \RU{она оставляет}\EN{it leaves} $a$ \InENRU \ST{0}.

\RU{Последняя}\EN{The last} \FSTP \RU{оставляет содержимое}\EN{leaves} \ST{0} 
\RU{на вершине стека, выбрасывая содержимое}\EN{on top of the stack, discarding the contents of} \ST{1}.

\ifdefined\IncludeGDB
\RU{Попробуем оттрасировать функцию в}\EN{Let's trace this function in} GDB:

\lstinputlisting[caption=\Optimizing GCC 4.8.1 and GDB,numbers=left]{patterns/12_FPU/3_comparison/x86/gdb.txt}

\RU{Используя}\EN{Using} \q{ni}, \RU{дадим первым двум инструкциям \FLD исполниться.}
\EN{let's execute the first two \FLD instructions.}

\RU{Посмотрим регистры FPU}\EN{Let's examine the FPU registers} (\LineENRU 33).

\RU{Как уже было указано ранее, регистры FPU это скорее кольцевой буфер, нежели стек}%
\EN{As it was mentioned before, the FPU registers set is a circular buffer rather than a stack} (\myref{FPU_is_rather_circular_buffer}).
\RU{И}\EN{And} GDB \RU{показывает не регистры}\EN{doesn't show} \TT{STx}\RU{, а внутренние регистры FPU}
\EN{registers, but internal the FPU registers} (\TT{Rx}). 
\RU{Стрелка}\EN{The arrow} (\RU{на строке}\EN{at line} 35) 
\RU{указывает на текущую вершину стека.}
\EN{points to the current top of the stack.}
\RU{Вы можете также увидеть содержимое регистра \TT{TOP} в \q{Status Word} (строка 44). Там сейчас 6, так что
вершина стека сейчас указывает на внутренний регистр 6.}
\EN{You can also see the \TT{TOP} register contents in \IT{Status Word} (line 44)---it is 6 now, 
so the stack top is now pointing to internal register 6.}

\RU{Значения $a$ и $b$ меняются местами после исполнения \TT{FXCH}}\EN{The values of $a$ and $b$ are swapped 
after \TT{FXCH} is executed} (\LineENRU 54).

\TT{FUCOMI} \RU{исполнилась}\EN{is executed} (\LineENRU 83). 
\RU{Посмотрим флаги}\EN{Let's see the flags}: \CF \RU{выставлен}\EN{is set} (\LineENRU 95).

\TT{FCMOVBE} \RU{действительно скопировал значение $b$ (см. строку 104).}
\EN{has copied the value of $b$ (see line 104).}

\FSTP \RU{оставляет одно значение на вершине стека}\EN{leaves one value at the top of stack} (\LineENRU 136). 
\RU{Значение \TT{TOP} теперь 7, так что вершина FPU-стека указывает на внутренний регистр 7}\EN{The value of \TT{TOP} is 
now 7, so the FPU stack top is pointing to internal register 7}.
\fi
\fi


\ifdefined\IncludeARM
\subsection{ARM}

\subsubsection{\OptimizingXcodeIV (\ARMMode)}

\lstinputlisting[caption=\OptimizingXcodeIV (\ARMMode)]{patterns/12_FPU/3_comparison/ARM/Xcode_ARM.lst.\LANG}

\index{ARM!\Registers!APSR}
\index{ARM!\Registers!FPSCR}
\RU{Очень простой случай.}\EN{A very simple case.}
\RU{Входные величины помещаются в}\EN{The input values are placed into the} \TT{D17} \AndENRU \TT{D16} 
\RU{и сравниваются при помощи инструкции}\EN{registers and then compared using the} 
\TT{VCMPE}\EN{ instruction}.
\RU{Как и в сопроцессорах x86, сопроцессор в ARM имеет свой собственный регистр статуса и флагов}%
\EN{Just like in the x86 coprocessor, the ARM coprocessor has its own status and flags register} (\ac{FPSCR}),
\RU{потому что есть необходимость хранить специфичные для его работы флаги.}
\EN{since there is a need to store coprocessor-specific flags.}
% TODO -> расписать регистр по битам
\index{ARM!\Instructions!VMRS}
\RU{И так же, как и в x86}\EN{And just like in x86}, 
\RU{в ARM нет инструкций условного перехода}%
\EN{there are no conditional jump instruction in ARM}, 
\RU{проверяющих биты в регистре статуса сопроцессора}\EN{that can check bits in the status register of the coprocessor}. 
\RU{Поэтому имеется инструкция}\EN{So there is} \TT{VMRS}%
\RU{, копирующая 4 бита}\EN{, which copies 4 bits} (N, Z, C, V) 
\RU{из статуса сопроцессора в биты \IT{общего} статуса (регистр \ac{APSR}).}
\EN{from the coprocessor status word into bits of the \IT{general} status register (\ac{APSR}).}

\index{ARM!\Instructions!VMOVGT}
\TT{VMOVGT} \RU{это аналог}\EN{is the analog of the} \TT{MOVGT}, 
\RU{инструкция для D-регистров, срабатывающая, если при сравнении один операнд был больше чем второй}
\EN{instruction for D-registers, it executes if one operand is greater than the other while comparing} 
(\IT{GT\EMDASH{}Greater Than}). 

\RU{Если она сработает}\EN{If it gets executed}, 
\RU{в \TT{D16} запишется значение $b$}\EN{the value of $b$ is to be written into \TT{D16}}%
\RU{, лежащее в тот момент в}\EN{(that is currently stored in in} \TT{D17}\EN{)}.

\RU{В обратном случае}\EN{Otherwise} 
\RU{в \TT{D16} остается значение $a$.}
\EN{the value of $a$ stays in the \TT{D16} register.}

\index{ARM!\Instructions!VMOV}
\RU{Предпоследняя инструкция \TT{VMOV} готовит то, что было в \TT{D16}, для возврата через 
пару регистров \Reg{0} и \Reg{1}.}
\EN{The penultimate instruction \TT{VMOV} prepares the value in the \TT{D16} register for returning it via the \Reg{0} and \Reg{1}
register pair.}

\subsubsection{\OptimizingXcodeIV (\ThumbTwoMode)}

\begin{lstlisting}[caption=\OptimizingXcodeIV (\ThumbTwoMode)]
VMOV            D16, R2, R3 ; b
VMOV            D17, R0, R1 ; a
VCMPE.F64       D17, D16
VMRS            APSR_nzcv, FPSCR
IT GT 
VMOVGT.F64      D16, D17
VMOV            R0, R1, D16
BX              LR
\end{lstlisting}

\RU{Почти то же самое, что и в предыдущем примере, за парой отличий.}
\EN{Almost the same as in the previous example, however slightly different.}
\RU{Как мы уже знаем, многие инструкции в режиме ARM можно дополнять условием.}
\EN{As we already know, many instructions in ARM mode can be supplemented by condition predicate.}

\RU{Но в режиме Thumb такого нет.}
\EN{But there is no such thing in Thumb mode.} 
\RU{В 16-битных инструкций просто нет места для лишних 4 битов, при помощи
которых можно было бы закодировать условие выполнения.}
\EN{There is no space in the 16-bit instructions for 4 more bits in which conditions can be encoded.}

\index{ARM!\ThumbTwoMode}
\RU{Поэтому в Thumb-2 добавили возможность дополнять Thumb-инструкции условиями.}
\EN{However, Thumb-2 was extended to make it possible to specify predicates to old Thumb instructions.}

\RU{В листинге, сгенерированном при помощи \IDA, мы видим инструкцию \TT{VMOVGT}, 
такую же как и в предыдущем примере.}
\EN{Here, in the \IDA-generated listing, we see the \TT{VMOVGT} instruction, as in previous example.}

\RU{В реальности}\EN{In fact,} 
\RU{там закодирована обычная инструкция \TT{VMOV}}%
\EN{the usual \TT{VMOV} is encoded there}, 
\RU{просто \IDA добавила суффикс \TT{-GT} к ней}%
\EN{but \IDA adds the \TT{-GT} suffix to it}, 
\RU{потому что перед этой инструкцией стоит \TT{IT GT}.}
\EN{since there is a \TT{\q{IT GT}} instruction placed right before it.}

\label{ARM_Thumb_IT}
\index{ARM!\Instructions!IT}
\index{ARM!if-then block}
\EN{The}\RU{Инструкция} \TT{IT} \RU{определяет так называемый}\EN{instruction defines a so-called} \IT{if-then block}. 
\RU{После этой инструкции можно указывать до четырех инструкций, 
к каждой из которых будет добавлен суффикс условия.}
\EN{After the instruction it is possible to place up to 4 instructions, 
each of them has a predicate suffix.}
\RU{В нашем примере}\EN{In our example,} \TT{IT GT} \RU{означает,}\EN{implies}
\RU{что следующая за ней инструкция будет исполнена, если условие}
\EN{that the next instruction is to be executed, if the}
\IT{GT} (\IT{Greater Than}) \RU{справедливо}\EN{condition is true}.

\index{Angry Birds}
\RU{Теперь более сложный пример. Кстати, из}\EN{Here is a more complex code fragment, by the way, from} 
Angry Birds (\RU{для}\EN{for} iOS):

% FIXME russian listing:
\begin{lstlisting}[caption=Angry Birds Classic]
...
ITE NE
VMOVNE          R2, R3, D16
VMOVEQ          R2, R3, D17
BLX             _objc_msgSend ; not prefixed
...
\end{lstlisting}

\TT{ITE} \RU{означает}\EN{stands for} \IT{if-then-else} 
\RU{и кодирует суффиксы для двух следующих за ней инструкций.}
\EN{and it encodes suffixes for the next two instructions.}
\RU{Первая из них исполнится, если условие, закодированное в}
\EN{The first instruction executes if the condition encoded in} \TT{ITE} (\IT{NE, not equal}) 
\RU{будет в тот момент справедливо}\EN{is true at},
\RU{а вторая~--- если это условие не сработает}\EN{and the 
second---if the condition is not true}.
(\RU{Обратное условие от}\EN{The inverse condition of} \TT{NE} \RU{это}\EN{is} \TT{EQ} (\IT{equal})).

\EN{The instruction followed after the second VMOV (or VMOVEQ) is a normal one, not prefixed (BLX).}
\RU{Инструкция следующая за второй VMOV (или VMOEQ) нормальная, без префикса (BLX).}

\index{Angry Birds}
\RU{Ещё чуть сложнее}\EN{One more that's slightly harder}, 
\RU{и снова этот фрагмент из}\EN{which is also from} Angry Birds:

% FIXME russian listing:
\begin{lstlisting}[caption=Angry Birds Classic]
...
ITTTT EQ
MOVEQ           R0, R4
ADDEQ           SP, SP, #0x20
POPEQ.W         {R8,R10}
POPEQ           {R4-R7,PC}
BLX             ___stack_chk_fail ; not prefixed
...
\end{lstlisting}

\RU{Четыре символа \q{T} в инструкции означают, что четыре последующие инструкции будут исполнены если условие соблюдается.}
\EN{Four \q{T} symbols in the instruction mnemonic mean 
that the four subsequent instructions are to be executed if the condition is true.}
\RU{Поэтому \IDA добавила ко всем четырем инструкциям суффикс}
\EN{That's why \IDA adds the} \TT{-EQ}\EN{ suffix
to each one of them}. 

\RU{А если бы здесь было, например,}\EN{And if there was be, for example,}
\TT{ITEEE EQ} (\IT{if-then-else-else-else}), 
\RU{тогда суффиксы для следующих четырех инструкций были бы расставлены так:}
\EN{then the suffixes would have been set as follows:}

\begin{lstlisting}
-EQ
-NE
-NE
-NE
\end{lstlisting}

\index{Angry Birds}
\RU{Ещё фрагмент из}\EN{Another fragment from} Angry Birds:

% FIXME russian listing:
\begin{lstlisting}[caption=Angry Birds Classic]
...
CMP.W           R0, #0xFFFFFFFF
ITTE LE
SUBLE.W         R10, R0, #1
NEGLE           R0, R0
MOVGT           R10, R0
MOVS            R6, #0         ; not prefixed
CBZ             R0, loc_1E7E32 ; not prefixed
...
\end{lstlisting}

\TT{ITTE} (\IT{if-then-then-else}) 
\RU{означает, что первая и вторая инструкции исполнятся, если условие \TT{LE} (\IT{Less or Equal})
справедливо, а третья~--- если справедливо обратное условие (\TT{GT}\EMDASH\IT{Greater Than}).}
\EN{implies that the 1st and 2nd instructions are to be executed if the \TT{LE} (\IT{Less or Equal})
condition is true, and the 3rd---if the inverse condition (\TT{GT}\EMDASH\IT{Greater Than}) 
is true.}

\RU{Компиляторы способны генерировать далеко не все варианты.}
\EN{Compilers usually don't generate all possible combinations.}
\index{Angry Birds}
\RU{Например, в вышеупомянутой игре Angry Birds (версия \IT{classic} для iOS)}
\EN{For example, in the mentioned Angry Birds game (\IT{classic} version for iOS)}
\RU{встречаются только такие варианты инструкции \TT{IT}}\EN{only these variants of the \TT{IT} instruction are used}: 
\TT{IT}, \TT{ITE}, \TT{ITT}, \TT{ITTE}, \TT{ITTT}, \TT{ITTTT}.
\index{\GrepUsage}
\RU{Как это узнать?}\EN{How to learn this?}
\RU{В \IDA можно сгенерировать листинг (что и было сделано), только в опциях был установлен показ 4 байтов для каждого опкода.}
\EN{In \IDA It is possible to produce listing files, so it was created with an option to show 4 bytes for each opcode.}
\RU{Затем, зная что старшая часть 16-битного опкода (\TT{IT} это \TT{0xBF}), сделаем при помощи \TT{grep} это:}
\EN{Then, knowing the high part of the 16-bit opcode (\TT{IT} is \TT{0xBF}), we do the following using \TT{grep}:}

\begin{lstlisting}
cat AngryBirdsClassic.lst | grep " BF" | grep "IT" > results.lst
\end{lstlisting}

\index{ARM!\ThumbTwoMode}
\RU{Кстати, если писать на ассемблере для режима Thumb-2 вручную, и дополнять инструкции суффиксами
условия, то ассемблер автоматически будет добавлять инструкцию \TT{IT} с соответствующими флагами там,
где надо.}
\EN{By the way, if you program in ARM assembly language manually for Thumb-2 mode, 
and you add conditional suffixes,
the assembler will add the \TT{IT} instructions automatically with the required flags where it is necessary.}

\subsubsection{\NonOptimizingXcodeIV (\ARMMode)}

\begin{lstlisting}[caption=\NonOptimizingXcodeIV (\ARMMode)]
b               = -0x20
a               = -0x18
val_to_return   = -0x10
saved_R7        = -4

                STR             R7, [SP,#saved_R7]!
                MOV             R7, SP
                SUB             SP, SP, #0x1C
                BIC             SP, SP, #7
                VMOV            D16, R2, R3
                VMOV            D17, R0, R1
                VSTR            D17, [SP,#0x20+a]
                VSTR            D16, [SP,#0x20+b]
                VLDR            D16, [SP,#0x20+a]
                VLDR            D17, [SP,#0x20+b]
                VCMPE.F64       D16, D17
                VMRS            APSR_nzcv, FPSCR
                BLE             loc_2E08
                VLDR            D16, [SP,#0x20+a]
                VSTR            D16, [SP,#0x20+val_to_return]
                B               loc_2E10

loc_2E08
                VLDR            D16, [SP,#0x20+b]
                VSTR            D16, [SP,#0x20+val_to_return]

loc_2E10
                VLDR            D16, [SP,#0x20+val_to_return]
                VMOV            R0, R1, D16
                MOV             SP, R7
                LDR             R7, [SP+0x20+b],#4
                BX              LR
\end{lstlisting}

\RU{Почти то же самое, что мы уже видели}\EN{Almost the same as we already saw}, 
\RU{но много избыточного кода из-за хранения $a$ и $b$, 
а также выходного значения, в локальном стеке.}
\EN{but there is too much redundant code because the $a$ and $b$ variables are stored in the local stack, as well
as the return value.}

\subsubsection{\OptimizingKeilVI (\ThumbMode)}

\begin{lstlisting}[caption=\OptimizingKeilVI (\ThumbMode)]
                PUSH    {R3-R7,LR}
                MOVS    R4, R2
                MOVS    R5, R3
                MOVS    R6, R0
                MOVS    R7, R1
                BL      __aeabi_cdrcmple
                BCS     loc_1C0
                MOVS    R0, R6
                MOVS    R1, R7
                POP     {R3-R7,PC}

loc_1C0
                MOVS    R0, R4
                MOVS    R1, R5
                POP     {R3-R7,PC}
\end{lstlisting}

\RU{Keil не генерирует FPU-инструкции, потому что не 
рассчитывает на то, что они будет поддерживаться, а простым сравнением побитово здесь не обойтись.}
\EN{Keil doesn't generate FPU-instructions since it cannot rely on them being
supported on the target CPU, and it cannot be done by straightforward bitwise comparing.}
%TODO1: why?
\RU{Для сравнения вызывается библиотечная функция}\EN{So it calls an external library
function to do the comparison:} \TT{\_\_aeabi\_cdrcmple}. 
\index{ARM!\Instructions!BCS}\\
\\
N.B. \RU{Результат
сравнения эта функция оставляет в флагах, чтобы следующая за вызовом инструкция}
\EN{The result of the comparison is to be left in the flags by this function, so the following}
\TT{BCS} (\IT{Carry set\RU{~}---\RU{ }Greater than or equal})
\RU{могла работать без дополнительного кода.}\EN{instruction can work without any additional code.}

\subsection{ARM64}

\subsubsection{\Optimizing GCC (Linaro) 4.9}

\lstinputlisting{patterns/12_FPU/3_comparison/ARM/ARM64_GCC_O3.lst.\LANG}

\EN{The}\RU{В} ARM64 \ac{ISA} \RU{теперь есть FPU-инструкции, устанавливающие флаги CPU}\EN{has FPU-instructions 
which set} \ac{APSR} \RU{вместо}\EN{the CPU flags instead of} \ac{FPSCR} \EN{for convenience}\RU{для удобства}.
\EN{The}\ac{FPU} \RU{больше не отдельное устройство (по крайней мере логически)}\EN{is not a separate device here 
anymore (at least, logically)}.
\index{ARM!\Instructions!FCMPE}
\RU{Это}\EN{Here we see} \TT{FCMPE}. \RU{Она сравнивает два значения, переданных в}\EN{It compares the two values 
passed in} \RegD{0} \AndENRU \RegD{1} 
(\RU{а это первый и второй аргументы функции}\EN{which are the first and second arguments of the function})
\RU{и выставляет флаги в}\EN{and sets} \ac{APSR}\EN{ flags} (N, Z, C, V).

\index{ARM!\Instructions!FCSEL}
\TT{FCSEL} (\IT{Floating Conditional Select}) \RU{копирует значение}\EN{copies the value of} \RegD{0} \OrENRU 
\RegD{1} \RU{в}\EN{into} \RegD{0} \RU{в зависимости от условия}\EN{depending on the condition} 
(\TT{GT}\EMDASH{}Greater Than\RU{\EMDASH{}больше чем}),
\RU{и снова, она использует флаги в регистре}\EN{and again, it uses flags in} \ac{APSR} \RU{вместо}\EN{register
instead of} \ac{FPSCR}.
\RU{Это куда удобнее, если сравнивать с тем набором инструкций, что был в процессорах раньше.}
\EN{This is much more convenient, compared to the instruction set in older CPUs.}

\RU{Если условие верно}\EN{If the condition is true} (\TT{GT}), \RU{тогда значение из}\EN{then the value of} \RegD{0} 
\RU{копируется в}\EN{is copied into} \RegD{0} (\RU{т.е. ничего не происходит}\EN{i.e., nothing happens}).
\RU{Если условие не верно, то значение}\EN{If the condition is not true, the value of} \RegD{1} 
\RU{копируется в}\EN{is copied into} \RegD{0}.

\subsubsection{\NonOptimizing GCC (Linaro) 4.9}

\lstinputlisting{patterns/12_FPU/3_comparison/ARM/ARM64_GCC.lst.\LANG}

\RU{Неоптимизирующий GCC более многословен}\EN{Non-optimizing GCC is more verbose}.
\RU{В начале функция сохраняет значения входных аргументов в локальном стеке}
\EN{First, the function saves its input argument values in the local stack} (\IT{Register Save Area}).
\RU{Затем код перезагружает значения в регистры}\EN{Then the code reloads these values into registers}
\RegX{0}/\RegX{1} \RU{и наконец копирует их в}\EN{and finally copies them to} 
\RegD{0}/\RegD{1} \RU{для сравнения инструкцией}\EN{to be compared using} \TT{FCMPE}. 
\RU{Много избыточного кода, но так работают неоптимизирующие компиляторы}\EN{A lot of redundant code, 
but that is how non-optimizing compilers work}.
\TT{FCMPE} \RU{сравнивает значения и устанавливает флаги в}\EN{compares the values and sets the} \ac{APSR}\EN{ flags}.
\RU{В этот момент компилятор ещё не думает о более удобной инструкции}\EN{At this moment, 
the compiler is not thinking yet about the more convenient} \TT{FCSEL}\RU{, так что он работает старым 
методом}\EN{ instruction, so it proceed using old methods}: 
\RU{использует инструкцию}\EN{using the} \TT{BLE}\EN{ instruction} (\IT{Branch if Less than or Equal}\RU{ (переход
если меньше или равно)}).
\RU{В одном случае}\EN{In the first case} ($a>b$)\EN{, the value of}\RU{ значение} $a$ \RU{перезагружается в}\EN{gets loaded 
into} \RegX{0}.
\RU{В другом случае}\EN{In the other case} ($a<=b$)\EN{, the value of}\RU{ значение} $b$ \RU{загружается в}\EN{gets loaded into} 
\RegX{0}.
\RU{Наконец, значение из}\EN{Finally, the value from} \RegX{0} \RU{копируется в}\EN{gets copied into} \RegD{0}, 
\RU{потому что возвращаемое значение оставляется в этом регистре}\EN{because the return value needs to be in this 
register}.

\myparagraph{\Exercise}

\RU{Для упражнения вы можете попробовать оптимизировать этот фрагмент кода вручную, удалив избыточные инструкции,
но не добавляя новых (включая \TT{FCSEL})}\EN{As an exercise, you can try optimizing this piece of code 
manually by removing redundant instructions and not introducing new ones (including \TT{FCSEL})}.

\subsubsection{\Optimizing GCC (Linaro) 4.9\EMDASH{}float}

\RU{Перепишем пример. Теперь здесь \Tfloat вместо \Tdouble.}%
\EN{Let's also rewrite this example to use \Tfloat instead of \Tdouble.}

\begin{lstlisting}
float f_max (float a, float b)
{
	if (a>b)
		return a;

	return b;
};
\end{lstlisting}

\lstinputlisting{patterns/12_FPU/3_comparison/ARM/ARM64_GCC_O3_float.lst.\LANG}

\RU{Всё то же самое, только используются S-регистры вместо D-.}
\EN{It is the same code, but the S-registers are used instead of D- ones.}
\RU{Так что числа типа \Tfloat передаются в 32-битных S-регистрах (а это младшие части 64-битных D-регистров).}
\EN{It's because numbers of type \Tfloat are passed in 32-bit S-registers (which are in fact the lower parts of the 64-bit D-registers).}


\fi
\ifdefined\IncludeMIPS
\subsection{MIPS}

\index{MIPS!\Registers!FCCR}
\EN{The co-processor of the MIPS processor has a condition bit which can be set in the FPU and checked in the CPU.}
\RU{В сопроцессоре MIPS есть бит результата, который устанавливается в FPU и проверяется в CPU.}
\EN{Earlier MIPS-es have only one condition bit (called FCC0), later ones have 8 (called FCC7-FCC0).}
\RU{Ранние MIPS имели только один бит (с названием FCC0), а у поздних их 8 (с названием FCC7-FCC0).}
\RU{Этот бит (или биты) находятся в регистре с названием FCCR.}
\EN{This bit (or bits) are located in the register called FCCR.}

\lstinputlisting[caption=\Optimizing GCC 4.4.5 (IDA)]{patterns/12_FPU/3_comparison/MIPS_O3_IDA.lst.\LANG}

\index{MIPS!\Instructions!C.LT.D}
\TT{C.LT.D} \EN{compares two values}\RU{сравнивает два значения}. 
\TT{LT} \EN{is the condition}\RU{это условие} \q{Less Than}\RU{ (меньше чем)}.
\TT{D} \EN{implies values of type}\RU{означает переменные типа} \Tdouble.
\EN{Depending on the result of the comparison, the FCC0 condition bit is either set or cleared.}
\RU{В зависимости от результата сравнения, бит FCC0 устанавливается или очищается.}

\index{MIPS!\Instructions!BC1T}
\index{MIPS!\Instructions!BC1F}
\TT{BC1T} \EN{checks the FCC0 bit and jumps if the bit is set}\RU{проверяет бит FCC0 и делает переход, если бит выставлен}.
\TT{T} \EN{mean that the jump is to be taken if the bit is set}\RU{означает что переход произойдет если бит выставлен} (\q{True}).
\EN{There is also the instruction}\RU{Имеется также инструкция} \q{BC1F} \EN{which jumps if the bit is cleared}\RU{которая сработает, если бит сброшен} (\q{False}).

\RU{В зависимости от перехода один из аргументов функции помещается в регистр \$F0.}
\EN{Depending on the jump, one of function arguments is placed into \$F0.}

\fi


\section{\RU{Стек, калькуляторы и обратная польская запись}\EN{Stack, calculators and reverse Polish notation}}

\index{\RU{Обратная польская запись}\EN{Reverse Polish notation}}
\RU{Теперь понятно, почему некоторые старые калькуляторы использовали обратную польскую запись%
\footnote{\href{http://go.yurichev.com/17355}{ru.wikipedia.org/wiki/Обратная\_польская\_запись}}.}
\EN{Now we undestand why some old calculators used reverse Polish notation
\footnote{\href{http://go.yurichev.com/17354}{wikipedia.org/wiki/Reverse\_Polish\_notation}}.}
\RU{Например для сложения 12 и 34 нужно было набрать 12, потом 34, потом нажать знак \q{плюс}.}
\EN{For example, for addition of 12 and 34 one has to enter 12, then 34, then press \q{plus} sign.}
\RU{Это потому что старые калькуляторы просто реализовали стековую машину и это было куда проще, 
чем обрабатывать сложные выражения со скобками.}
\EN{It's because old calculators were just stack machine implementations, and this was much simpler
than to handle complex parenthesized expressions.}
\section{x64}

\RU{О том, как происходит работа с числами с плавающей запятой в x86-64, читайте здесь: \myref{floating_SIMD}.}
\EN{On how floating point numbers are processed in x86-64, read more here: \myref{floating_SIMD}.}

% sections
\ifdefined\IncludeExercises
\section{\Exercises}

\subsection{\Exercise \#1}

\RU{Избавьтесь от инструкции}\EN{Eliminate the} FXCH \RU{в примере}\EN{instruction in the example} 
\myref{gcc481_o3} \RU{и протестируйте его}\EN{and test it}.

\subsection{\Exercise \#2}
\label{exercise_FPU_2}

\WhatThisCodeDoes\

\begin{lstlisting}[caption=\Optimizing MSVC 2010]
__real@4014000000000000 DQ 04014000000000000r	; 5

_a1$ = 8	; size = 8
_a2$ = 16	; size = 8
_a3$ = 24	; size = 8
_a4$ = 32	; size = 8
_a5$ = 40	; size = 8
_f	PROC
	fld	QWORD PTR _a1$[esp-4]
	fadd	QWORD PTR _a2$[esp-4]
	fadd	QWORD PTR _a3$[esp-4]
	fadd	QWORD PTR _a4$[esp-4]
	fadd	QWORD PTR _a5$[esp-4]
	fdiv	QWORD PTR __real@4014000000000000
	ret	0
_f	ENDP
\end{lstlisting}

\begin{lstlisting}[caption=\NonOptimizingKeilVI (\ThumbMode{} / \RU{скомпилировано для}\EN{compiled for} Cortex-R4F CPU)]
f PROC
        VADD.F64 d0,d0,d1
        VMOV.F64 d1,#5.00000000
        VADD.F64 d0,d0,d2
        VADD.F64 d0,d0,d3
        VADD.F64 d2,d0,d4
        VDIV.F64 d0,d2,d1
        BX       lr
        ENDP
\end{lstlisting}

\begin{lstlisting}[caption=\Optimizing GCC 4.9 (ARM64)]
f:
	fadd	d0, d0, d1
	fmov	d1, 5.0e+0
	fadd	d2, d0, d2
	fadd	d3, d2, d3
	fadd	d0, d3, d4
	fdiv	d0, d0, d1
	ret
\end{lstlisting}

\lstinputlisting[caption=\Optimizing GCC 4.4.5 (MIPS) (IDA)]{patterns/12_FPU/ex_MIPS_O3_IDA.lst}

\Answer\: \myref{exercise_solutions_FPU_2}.

\fi
