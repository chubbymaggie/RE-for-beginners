\section{\RU{Уровень}\EN{Level} 2}

\subsection{\Exercise 2.1}

\EN{This function is calculates a square root by Newton's method, the algorithm is taken 
from \cite{SICP}.}
\RU{Это функция вычисления квадратного корня методом Ньютона, алгоритм взят из \cite{SICP}.}

\Sourcecode: \href{http://go.yurichev.com/17148}{beginners.re}

% 2.2
% 2.3

\subsection{\Exercise 2.4}

\index{\CStandardLibrary!strstr()}
\RU{Ответ}\EN{Solution}: \TT{strstr()}.

\RU{Исходник на Си}\EN{C source code}:

\begin{lstlisting}
char * strstr (
        const char * str1,
        const char * str2
        )
{
        char *cp = (char *) str1;
        char *s1, *s2;

        if ( !*str2 )
            return((char *)str1);

        while (*cp)
        {
                s1 = cp;
                s2 = (char *) str2;

                while ( *s1 && *s2 && !(*s1-*s2) )
                        s1++, s2++;

                if (!*s2)
                        return(cp);

                cp++;
        }

        return(NULL);

}
\end{lstlisting}

% 2.5

\subsection{\Exercise 2.6}

\RU{Подсказка: если погуглить применяемую здесь константу, это может помочь.}
\EN{Hint: it might be helpful to google the constant used here.}

\RU{Ответ: шифрование алгоритмом \ac{TEA}}\EN{Solution: \ac{TEA} encryption algorithm}.

\RU{Исходник на Си}\EN{C source code} (\RU{взято с}\EN{taken from} \href{http://go.yurichev.com/17106}{wikipedia}):

\begin{lstlisting}
void f (unsigned int* v, unsigned int* k) {
    unsigned int v0=v[0], v1=v[1], sum=0, i;           /* set up */
    unsigned int delta=0x9e3779b9;                     /* a key schedule constant */
    unsigned int k0=k[0], k1=k[1], k2=k[2], k3=k[3];   /* cache key */
    for (i=0; i < 32; i++) {                       /* basic cycle start */
        sum += delta;
        v0 += ((v1<<4) + k0) ^ (v1 + sum) ^ ((v1>>5) + k1);
        v1 += ((v0<<4) + k2) ^ (v0 + sum) ^ ((v0>>5) + k3);  
    }                                              /* end cycle */
    v[0]=v0; v[1]=v1;
}
\end{lstlisting}

% 2.7
% 2.8
% 2.9
% 2.10
% 2.11
% 2.12

\subsection{\Exercise 2.13}

\RU{Это криптоалгоритм}\EN{The cryptographic algorithm is a} 
linear feedback shift register\RU{ (регистр сдвига с линейной обратной связью)}
\footnote{\href{http://go.yurichev.com/17000}{wikipedia}}.

\Sourcecode: \href{http://go.yurichev.com/17149}{beginners.re}

\subsection{\Exercise 2.14}

\RU{Это алгоритм нахождения наименьшего общего делителя}\EN{This is an algorithm for finding the greater
common divisor (GCD)}.

\Sourcecode: \href{http://go.yurichev.com/17150}{beginners.re}

\subsection{\Exercise 2.15}

\RU{Это вычисление числа Пи методом Монте-Карло}\EN{Pi value calculation using the Monte-Carlo method}.

\Sourcecode: \href{http://go.yurichev.com/17151}{beginners.re}

\subsection{\Exercise 2.16}

\RU{Это функция Аккермана}\EN{It is the Ackermann function}
\footnote{\href{http://go.yurichev.com/17001}{wikipedia}}.

\begin{lstlisting}
int ack (int m, int n)
{
	if (m==0)
		return n+1;
	if (n==0)
		return ack (m-1, 1);
	return ack(m-1, ack (m, n-1));
};
\end{lstlisting}

\subsection{\Exercise 2.17}

\RU{Это одномерный клеточный автомат по \IT{правилу 110}}\EN{This is 1D cellular automaton working
by \IT{Rule 110}}:\\
\href{http://go.yurichev.com/17002}{wikipedia}.

\Sourcecode: \href{http://go.yurichev.com/17152}{beginners.re}

\subsection{\Exercise 2.18}

\Sourcecode: \href{http://go.yurichev.com/17153}{beginners.re}

\subsection{\Exercise 2.19}

\Sourcecode: \href{http://go.yurichev.com/17154}{beginners.re}

\subsection{\Exercise 2.20}

\RU{Подсказка}\EN{Hint}: the On-Line Encyclopedia of Integer Sequences (OEIS) \RU{(энциклопедия целочисленных последовательностей}) \RU{может помочь}\EN{may help}.

\RU{Ответ}\EN{Answer}: \RU{это}\EN{these are} \IT{hailstone numbers}, \RU{имеющие отношения к гипотезе Коллатца}\EN{which relate to the Collatz conjecture}\footnote{\EN{\href{http://go.yurichev.com/17107}{wikipedia}}\RU{\href{http://go.yurichev.com/17108}{wikipedia}}}.

\Sourcecode: \href{http://go.yurichev.com/17155}{beginners.re}

