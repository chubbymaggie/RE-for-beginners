\index{x86!\Instructions!LEA}
\item[LEA] (\IT{Load Effective Address}) \RU{сформировать адрес}\EN{form an address}

\label{sec:LEA}

\newcommand{\URLAM}{\href{http://go.yurichev.com/17109}{wikipedia}}

\RU{Это инструкция, которая задумывалась вовсе не для складывания 
и умножения чисел, 
а для формирования адреса например, из указателя на массив и прибавления индекса к нему
\footnote{См. также: \URLAM}.}
\EN{This instruction was intended not for summing values and multiplication 
but for forming an address, 
e.g., for calculating the address of an array element by adding the array address, element index, with 
multiplication of element size\footnote{See also: \URLAM}.}\ESph{}\PTBRph{}\PLph{} \\
\\
\RU{То есть, разница между \MOV и \LEA в том, что \MOV формирует адрес в памяти 
и загружает значение из памяти, либо записывает его туда, а \LEA только формирует адрес.}
\EN{So, the difference between \MOV and \LEA is that \MOV forms a memory address and loads a value
from memory or stores it there, but \LEA just forms an address.}\ESph{}\PTBRph{}\PLph{} \\
\\
\RU{Тем не менее, её можно использовать для любых других вычислений}
\EN{But nevertheless, it is can be used for any other calculations}\ESph{}\PTBRph{}\PLph{}. \\
\\
\RU{\LEA удобна тем, что производимые ею вычисления не модифицируют флаги \ac{CPU}. 
Это может быть очень важно для \ac{OOE} процессоров (чтобы было меньше зависимостей между данными).}
\EN{\LEA is convenient because the computations performed by it does not alter \ac{CPU} flags.
This may be very important for \ac{OOE} processors (to create less data dependencies).}

\begin{lstlisting}
int f(int a, int b)
{
	return a*8+b;
};
\end{lstlisting}

\begin{lstlisting}[caption=\Optimizing MSVC 2010]
_a$ = 8							; size = 4
_b$ = 12						; size = 4
_f	PROC
	mov	eax, DWORD PTR _b$[esp-4]
	mov	ecx, DWORD PTR _a$[esp-4]
	lea	eax, DWORD PTR [eax+ecx*8]
	ret	0
_f	ENDP
\end{lstlisting}

\index{Intel C++}
Intel C++ \RU{использует LEA даже больше}\EN{uses LEA even more}:

\begin{lstlisting}
int f1(int a)
{
	return a*13;
};
\end{lstlisting}

\begin{lstlisting}[caption=Intel C++ 2011]
_f1	PROC NEAR 
        mov       ecx, DWORD PTR [4+esp]      ; ecx = a
	lea       edx, DWORD PTR [ecx+ecx*8]  ; edx = a*9
	lea       eax, DWORD PTR [edx+ecx*4]  ; eax = a*9 + a*4 = a*13
        ret                                
\end{lstlisting}

\RU{Эти две инструкции вместо одной IMUL будут работать быстрее.}
\EN{These two instructions performs faster than one IMUL.}

