\chapter{\RU{Обрезка строк}\EN{Strings trimming}}
\newcommand{\CRLF}{\ac{CR}/\ac{LF}}

\RU{Весьма востребованная операция со строками\EMDASH{}это удаление некоторых символов в начале и/или конце
строки.}
\EN{A very common string processing task is to remove some characters at the start and/or at the end.}

\RU{В этом примере, мы будем работать с функцией, удаляющей все символы перевода строки 
(\CRLF{}) в конце входной строки:}
\EN{In this example, we are going to work with a function which removes all newline characters 
(\CRLF{}) from the end of the input string:}

\lstinputlisting{\CURPATH/strtrim.c.\LANG}

\RU{Входной аргумент всегда возвращается на выходе, это удобно, когда вам нужно объединять
функции обработки строк в цепочки, как это сделано здесь в функции \main.}
\EN{The input argument is always returned on exit, this is convenient when you need to chain 
string processing functions, like it was done here in the \main function.}

\RU{Вторая часть for() (\TT{str\_len>0 \&\& (c=s[str\_len-1])}) называется в \CCpp \q{short-circuit} 
(короткое замыкание) и это очень удобно: \cite[1.3.8]{CBook}.}
\EN{The second part of for() (\TT{str\_len>0 \&\& (c=s[str\_len-1])}) is the so called \q{short-circuit} 
in \CCpp and is very convenient \cite[1.3.8]{CBook}.}
\RU{Компиляторы \CCpp гарантируют последовательное вычисление слева направо.}
\EN{The \CCpp compilers guarantee an evaluation sequence from left to right.}
\RU{Так что если первое условие не истинно после вычисления, второе никогда не будет
вычисляться.}
\EN{So if the first clause is false after evaluation, the second one is never to be evaluated.}

% subsections
\subsection{x64: \RU{8 аргументов}\EN{8 arguments}}

\index{x86-64}
\label{example_printf8_x64}
\RU{Для того чтобы посмотреть, как остальные аргументы будут передаваться через стек, 
изменим пример ещё раз, 
увеличив количество передаваемых аргументов до 9 
(строка формата \printf и 8 переменных типа \Tint)}%
\EN{To see how other arguments are passed via the stack, let's change our example again 
by increasing the number of arguments to 9 (\printf format string + 8 \Tint variables)}:

\lstinputlisting{patterns/03_printf/2.c}

\subsubsection{MSVC}

\RU{Как уже было сказано ранее, первые 4 аргумента в Win64 передаются в регистрах}
\EN{As it was mentioned earlier, the first 4 arguments has to be passed through the} \RCX, \RDX, \Reg{8}, \Reg{9}
\RU{, а остальные~--- через стек}\EN{ registers in Win64, while all the rest---via the stack}.
\RU{Здесь мы это и видим}\EN{That is exactly what we see here}.
\RU{Впрочем, инструкция \PUSH не используется, вместо неё при помощи \MOV значения сразу записываются в стек}%
\EN{However, the \MOV instruction, instead of \PUSH, is used for preparing the stack, so the values are stored
to the stack in a straightforward manner}.

\lstinputlisting[caption=MSVC 2012 x64]{patterns/03_printf/x86/2_MSVC_x64.asm.\LANG}

\RU{Наблюдательный читатель может спросить, почему для значений типа \Tint отводится 8 байт,
ведь нужно только 4?}
\EN{The observant reader may ask why are 8 bytes allocated for \Tint values, when 4 is enough?}
\RU{Да, это нужно запомнить: для значений всех типов более коротких чем 64-бита, отводится 8 байт.}
\EN{Yes, one has to remember: 8 bytes are allocated for any data type shorter than 64 bits.}
\RU{Это сделано для удобства: так всегда легко рассчитать адрес того или иного аргумента.}
\EN{This is established for the convenience's sake: it makes it easy to calculate the address of arbitrary argument.}
\RU{К тому же, все они расположены по выровненным адресам в памяти.}
\EN{Besides, they are all located at aligned memory addresses.}
% also for local variables?
\RU{В 32-битных средах точно также: для всех типов резервируется 4 байта в стеке.}
\EN{It is the same in the 32-bit environments: 4 bytes are reserved for all data types.}

\ifdefined\IncludeGCC
\subsubsection{GCC}

\RU{В *NIX-системах для x86-64 ситуация похожая, вот только первые 6 аргументов передаются через}
\EN{The picture is similar for x86-64 *NIX OS-es, except that the first 6 arguments are passed through the} \RDI, \RSI,
\RDX, \RCX, \Reg{8}, \Reg{9}\EN{ registers}.
\RU{Остальные~--- через стек}\EN{All the rest---via the stack}.
\RU{GCC генерирует код, записывающий указатель на строку в \EDI вместо \RDI~--- 
это мы уже рассмотрели чуть раньше}\EN{GCC generates the code storing the string pointer into \EDI instead of \RDI{}---we noted that previously}: \myref{hw_EDI_instead_of_RDI}.

\RU{Почему перед вызовом \printf очищается регистр \EAX мы уже рассмотрели ранее}%
\EN{We also noted earlier that the \EAX register has been cleared before a \printf call}: \myref{SysVABI_input_EAX}.

\lstinputlisting[caption=\Optimizing GCC 4.4.6 x64]{patterns/03_printf/x86/2_GCC_x64.s.\LANG}

\ifdefined\IncludeGDB
\subsubsection{GCC + GDB}
\index{GDB}

\RU{Попробуем этот пример в}\EN{Let's try this example in} \ac{GDB}.

\begin{lstlisting}
$ gcc -g 2.c -o 2
\end{lstlisting}

\begin{lstlisting}
$ gdb 2
GNU gdb (GDB) 7.6.1-ubuntu
Copyright (C) 2013 Free Software Foundation, Inc.
License GPLv3+: GNU GPL version 3 or later <http://gnu.org/licenses/gpl.html>
This is free software: you are free to change and redistribute it.
There is NO WARRANTY, to the extent permitted by law.  Type "show copying"
and "show warranty" for details.
This GDB was configured as "x86_64-linux-gnu".
For bug reporting instructions, please see:
<http://www.gnu.org/software/gdb/bugs/>...
Reading symbols from /home/dennis/polygon/2...done.
\end{lstlisting}

\begin{lstlisting}[caption=\RU{ставим точку останова на \printf{,} запускаем}\EN{let's set the breakpoint to \printf{,} and run}]
(gdb) b printf
Breakpoint 1 at 0x400410
(gdb) run
Starting program: /home/dennis/polygon/2 

Breakpoint 1, __printf (format=0x400628 "a=%d; b=%d; c=%d; d=%d; e=%d; f=%d; g=%d; h=%d\n") at printf.c:29
29	printf.c: No such file or directory.
\end{lstlisting}

\RU{В регистрах}\EN{Registers} \RSI/\RDX/\RCX/\Reg{8}/\Reg{9} 
\RU{всё предсказуемо}\EN{have the expected values}.
\RU{А }\RIP \RU{содержит адрес самой первой инструкции функции}\EN{has the address of the very first instruction
of the} \printf\EN{ function}.

\begin{lstlisting}
(gdb) info registers
rax            0x0	0
rbx            0x0	0
rcx            0x3	3
rdx            0x2	2
rsi            0x1	1
rdi            0x400628	4195880
rbp            0x7fffffffdf60	0x7fffffffdf60
rsp            0x7fffffffdf38	0x7fffffffdf38
r8             0x4	4
r9             0x5	5
r10            0x7fffffffdce0	140737488346336
r11            0x7ffff7a65f60	140737348263776
r12            0x400440	4195392
r13            0x7fffffffe040	140737488347200
r14            0x0	0
r15            0x0	0
rip            0x7ffff7a65f60	0x7ffff7a65f60 <__printf>
...
\end{lstlisting}

\begin{lstlisting}[caption=\RU{смотрим на строку формата}\EN{let's inspect the format string}]
(gdb) x/s $rdi
0x400628:	"a=%d; b=%d; c=%d; d=%d; e=%d; f=%d; g=%d; h=%d\n"
\end{lstlisting}

\RU{Дампим стек на этот раз с командой x/g}\EN{Let's dump the stack with the x/g command this time}\EMDASH{}g 
\RU{означает}\EN{stands for} \IT{giant words}, \RU{т.е. 64-битные слова}\EN{i.e., 64-bit words}.

\begin{lstlisting}
(gdb) x/10g $rsp
0x7fffffffdf38:	0x0000000000400576	0x0000000000000006
0x7fffffffdf48:	0x0000000000000007	0x00007fff00000008
0x7fffffffdf58:	0x0000000000000000	0x0000000000000000
0x7fffffffdf68:	0x00007ffff7a33de5	0x0000000000000000
0x7fffffffdf78:	0x00007fffffffe048	0x0000000100000000
\end{lstlisting}

\RU{Самый первый элемент стека, как и в прошлый раз, это}\EN{The very first stack element, 
just like in the previous case, is the} \ac{RA}.
\RU{Через стек также передаются 3 значения}\EN{3 values are also passed through the stack}: 6, 7, 8.
\RU{Видно, что 8 передается с неочищенной старшей 32-битной частью}\EN{We also see that 8 is passed
with the high 32-bits not cleared}: \TT{0x00007fff00000008}.
\RU{Это нормально, ведь передаются числа типа \Tint, а они 32-битные}\EN{That's OK, because the values have
\Tint type, which is 32-bit}.
\RU{Так что в старшей части регистра или памяти стека остался \q{случайный мусор}}\EN{So, the high register
or stack element part may contain \q{random garbage}}.

\RU{\ac{GDB} показывает всю функцию \main, если попытаться посмотреть, куда вернется управление после исполнения \printf}%
\EN{If you take a look at where the control will return after the \printf execution,
\ac{GDB} will show the entire \main function}:

\begin{lstlisting}
(gdb) set disassembly-flavor intel
(gdb) disas 0x0000000000400576
Dump of assembler code for function main:
   0x000000000040052d <+0>:	push   rbp
   0x000000000040052e <+1>:	mov    rbp,rsp
   0x0000000000400531 <+4>:	sub    rsp,0x20
   0x0000000000400535 <+8>:	mov    DWORD PTR [rsp+0x10],0x8
   0x000000000040053d <+16>:	mov    DWORD PTR [rsp+0x8],0x7
   0x0000000000400545 <+24>:	mov    DWORD PTR [rsp],0x6
   0x000000000040054c <+31>:	mov    r9d,0x5
   0x0000000000400552 <+37>:	mov    r8d,0x4
   0x0000000000400558 <+43>:	mov    ecx,0x3
   0x000000000040055d <+48>:	mov    edx,0x2
   0x0000000000400562 <+53>:	mov    esi,0x1
   0x0000000000400567 <+58>:	mov    edi,0x400628
   0x000000000040056c <+63>:	mov    eax,0x0
   0x0000000000400571 <+68>:	call   0x400410 <printf@plt>
   0x0000000000400576 <+73>:	mov    eax,0x0
   0x000000000040057b <+78>:	leave  
   0x000000000040057c <+79>:	ret    
End of assembler dump.
\end{lstlisting}

\RU{Заканчиваем исполнение \printf, исполняем инструкцию обнуляющую \EAX, 
удостоверяемся что в регистре \EAX именно ноль}\EN{Let's finish executing \printf, execute the instruction
zeroing \EAX, and note that the \EAX register has a value of exactly zero}.
\RIP \RU{указывает сейчас на инструкцию}\EN{now points to the} \TT{LEAVE}\RU{, т.е. предпоследнюю в функции \main}
\EN{ instruction, i.e., the penultimate one in the \main function}.

\begin{lstlisting}
(gdb) finish
Run till exit from #0  __printf (format=0x400628 "a=%d; b=%d; c=%d; d=%d; e=%d; f=%d; g=%d; h=%d\n") at printf.c:29
a=1; b=2; c=3; d=4; e=5; f=6; g=7; h=8
main () at 2.c:6
6		return 0;
Value returned is $1 = 39
(gdb) next
7	};
(gdb) info registers
rax            0x0	0
rbx            0x0	0
rcx            0x26	38
rdx            0x7ffff7dd59f0	140737351866864
rsi            0x7fffffd9	2147483609
rdi            0x0	0
rbp            0x7fffffffdf60	0x7fffffffdf60
rsp            0x7fffffffdf40	0x7fffffffdf40
r8             0x7ffff7dd26a0	140737351853728
r9             0x7ffff7a60134	140737348239668
r10            0x7fffffffd5b0	140737488344496
r11            0x7ffff7a95900	140737348458752
r12            0x400440	4195392
r13            0x7fffffffe040	140737488347200
r14            0x0	0
r15            0x0	0
rip            0x40057b	0x40057b <main+78>
...
\end{lstlisting}
\fi
\fi

\ifdefined\IncludeARM
\subsection{ARM64}

\subsubsection{\Optimizing GCC (Linaro) 4.9}

\lstinputlisting{patterns/12_FPU/3_comparison/ARM/ARM64_GCC_O3.lst.\LANG}

\EN{The}\RU{В} ARM64 \ac{ISA} \RU{теперь есть FPU-инструкции, устанавливающие флаги CPU}\EN{has FPU-instructions 
which set} \ac{APSR} \RU{вместо}\EN{the CPU flags instead of} \ac{FPSCR} \EN{for convenience}\RU{для удобства}.
\EN{The}\ac{FPU} \RU{больше не отдельное устройство (по крайней мере логически)}\EN{is not a separate device here 
anymore (at least, logically)}.
\index{ARM!\Instructions!FCMPE}
\RU{Это}\EN{Here we see} \TT{FCMPE}. \RU{Она сравнивает два значения, переданных в}\EN{It compares the two values 
passed in} \RegD{0} \AndENRU \RegD{1} 
(\RU{а это первый и второй аргументы функции}\EN{which are the first and second arguments of the function})
\RU{и выставляет флаги в}\EN{and sets} \ac{APSR}\EN{ flags} (N, Z, C, V).

\index{ARM!\Instructions!FCSEL}
\TT{FCSEL} (\IT{Floating Conditional Select}) \RU{копирует значение}\EN{copies the value of} \RegD{0} \OrENRU 
\RegD{1} \RU{в}\EN{into} \RegD{0} \RU{в зависимости от условия}\EN{depending on the condition} 
(\TT{GT}\EMDASH{}Greater Than\RU{\EMDASH{}больше чем}),
\RU{и снова, она использует флаги в регистре}\EN{and again, it uses flags in} \ac{APSR} \RU{вместо}\EN{register
instead of} \ac{FPSCR}.
\RU{Это куда удобнее, если сравнивать с тем набором инструкций, что был в процессорах раньше.}
\EN{This is much more convenient, compared to the instruction set in older CPUs.}

\RU{Если условие верно}\EN{If the condition is true} (\TT{GT}), \RU{тогда значение из}\EN{then the value of} \RegD{0} 
\RU{копируется в}\EN{is copied into} \RegD{0} (\RU{т.е. ничего не происходит}\EN{i.e., nothing happens}).
\RU{Если условие не верно, то значение}\EN{If the condition is not true, the value of} \RegD{1} 
\RU{копируется в}\EN{is copied into} \RegD{0}.

\subsubsection{\NonOptimizing GCC (Linaro) 4.9}

\lstinputlisting{patterns/12_FPU/3_comparison/ARM/ARM64_GCC.lst.\LANG}

\RU{Неоптимизирующий GCC более многословен}\EN{Non-optimizing GCC is more verbose}.
\RU{В начале функция сохраняет значения входных аргументов в локальном стеке}
\EN{First, the function saves its input argument values in the local stack} (\IT{Register Save Area}).
\RU{Затем код перезагружает значения в регистры}\EN{Then the code reloads these values into registers}
\RegX{0}/\RegX{1} \RU{и наконец копирует их в}\EN{and finally copies them to} 
\RegD{0}/\RegD{1} \RU{для сравнения инструкцией}\EN{to be compared using} \TT{FCMPE}. 
\RU{Много избыточного кода, но так работают неоптимизирующие компиляторы}\EN{A lot of redundant code, 
but that is how non-optimizing compilers work}.
\TT{FCMPE} \RU{сравнивает значения и устанавливает флаги в}\EN{compares the values and sets the} \ac{APSR}\EN{ flags}.
\RU{В этот момент компилятор ещё не думает о более удобной инструкции}\EN{At this moment, 
the compiler is not thinking yet about the more convenient} \TT{FCSEL}\RU{, так что он работает старым 
методом}\EN{ instruction, so it proceed using old methods}: 
\RU{использует инструкцию}\EN{using the} \TT{BLE}\EN{ instruction} (\IT{Branch if Less than or Equal}\RU{ (переход
если меньше или равно)}).
\RU{В одном случае}\EN{In the first case} ($a>b$)\EN{, the value of}\RU{ значение} $a$ \RU{перезагружается в}\EN{gets loaded 
into} \RegX{0}.
\RU{В другом случае}\EN{In the other case} ($a<=b$)\EN{, the value of}\RU{ значение} $b$ \RU{загружается в}\EN{gets loaded into} 
\RegX{0}.
\RU{Наконец, значение из}\EN{Finally, the value from} \RegX{0} \RU{копируется в}\EN{gets copied into} \RegD{0}, 
\RU{потому что возвращаемое значение оставляется в этом регистре}\EN{because the return value needs to be in this 
register}.

\myparagraph{\Exercise}

\RU{Для упражнения вы можете попробовать оптимизировать этот фрагмент кода вручную, удалив избыточные инструкции,
но не добавляя новых (включая \TT{FCSEL})}\EN{As an exercise, you can try optimizing this piece of code 
manually by removing redundant instructions and not introducing new ones (including \TT{FCSEL})}.

\subsubsection{\Optimizing GCC (Linaro) 4.9\EMDASH{}float}

\RU{Перепишем пример. Теперь здесь \Tfloat вместо \Tdouble.}%
\EN{Let's also rewrite this example to use \Tfloat instead of \Tdouble.}

\begin{lstlisting}
float f_max (float a, float b)
{
	if (a>b)
		return a;

	return b;
};
\end{lstlisting}

\lstinputlisting{patterns/12_FPU/3_comparison/ARM/ARM64_GCC_O3_float.lst.\LANG}

\RU{Всё то же самое, только используются S-регистры вместо D-.}
\EN{It is the same code, but the S-registers are used instead of D- ones.}
\RU{Так что числа типа \Tfloat передаются в 32-битных S-регистрах (а это младшие части 64-битных D-регистров).}
\EN{It's because numbers of type \Tfloat are passed in 32-bit S-registers (which are in fact the lower parts of the 64-bit D-registers).}


\subsection{ARM}

\subsubsection{\NonOptimizingKeilVI (\ARMMode)}

\lstinputlisting{patterns/13_arrays/1_simple/simple_Keil_ARM_O0.asm.\LANG}

\RU{Тип \Tint требует 32 бита для хранения (или 4 байта),}
\EN{\Tint type requires 32 bits for storage (or 4 bytes),}
\RU{так что для хранения 20 переменных типа \Tint, нужно 80 (\TT{0x50}) байт.}
\EN{so to store 20 \Tint variables 80 (\TT{0x50}) bytes are needed.}
\RU{Поэтому инструкция}\EN{So that is why the} \INS{SUB SP, SP, \#0x50} 
\RU{в прологе функции выделяет в локальном стеке под массив именно столько места.}
\EN{instruction in the function's prologue allocates exactly this amount of space in the stack.}

\RU{И в первом и во втором цикле итератор цикла \var{i} будет постоянно находится в регистре \Reg{4}.}
\EN{In both the first and second loops, the loop iterator \var{i} is placed in the \Reg{4} register.}

\index{ARM!Optional operators!LSL}
\RU{Число, которое нужно записать в массив, вычисляется так: $i*2$, и это эквивалентно 
сдвигу на 1 бит влево, так что инструкция \INS{MOV R0, R4,LSL\#1} делает это.}
\EN{The number that is to be written into the array is calculated as $i*2$, which is effectively equivalent 
to shifting it left by one bit, so \INS{MOV R0, R4,LSL\#1} instruction does this.}

\index{ARM!\Instructions!STR}
\INS{STR R0, [SP,R4,LSL\#2]} \RU{записывает содержимое \Reg{0} в массив}\EN{writes the contents of \Reg{0} into the array}.
\RU{Указатель на элемент массива вычисляется так: \ac{SP} указывает на начало массива, \Reg{4} это $i$.}
\EN{Here is how a pointer to array element is calculated: \ac{SP} points to the start of the array, \Reg{4} is $i$.}
\RU{Так что сдвигаем $i$ на 2 бита влево, что эквивалентно умножению на 4 
(ведь каждый элемент массива занимает 4 байта) и прибавляем это к адресу начала массива.}
\EN{So shifting $i$ left by 2 bits is effectively equivalent to multiplication by 4
(since each array element has a size of 4 bytes) and then it's added to the address of the start of the array.}

\index{ARM!\Instructions!LDR}
\RU{Во втором цикле используется обратная инструкция \INS{LDR R2, [SP,R4,LSL\#2]}.
Она загружает из массива нужное значение и указатель на него вычисляется точно так же.}
\EN{The second loop has an inverse \INS{LDR R2, [SP,R4,LSL\#2]}
instruction. It loads the value we need from the array, and the pointer to it is calculated likewise.}

\subsubsection{\OptimizingKeilVI (\ThumbMode)}

\lstinputlisting{patterns/13_arrays/1_simple/simple_Keil_thumb_O3.asm.\LANG}

\RU{Код для Thumb очень похожий.}\EN{Thumb code is very similar.}
\index{ARM!\Instructions!LSLS}
\RU{В Thumb имеются отдельные инструкции для битовых сдвигов (как \TT{LSLS}), 
вычисляющие и число для записи в массив и адрес каждого элемента массива.}
\EN{Thumb mode has special instructions for bit shifting (like \TT{LSLS}),
which calculates the value to be written into the array and the address of each element in the array as well.}

\RU{Компилятор почему-то выделил в локальном стеке немного больше места, 
однако последние 4 байта не используются.}
\EN{The compiler allocates slightly more space in the local stack, however, the last 4 bytes are not used.}

\subsubsection{\NonOptimizing GCC 4.9.1 (ARM64)}

\lstinputlisting[caption=\NonOptimizing GCC 4.9.1 (ARM64)]{patterns/13_arrays/1_simple/ARM64_GCC491_O0.s.\LANG}

\fi
\ifdefined\IncludeMIPS
\section{MIPS}

\EN{For some reason, optimizing GCC 4.4.5 generate just a division instruction:}
\RU{По какой-то причине, оптимизирующий GCC 4.4.5 сгенерировал просто инструкцию деления:}

\lstinputlisting[caption=\Optimizing GCC 4.4.5 (IDA)]{\CURPATH/MIPS_O3_IDA.lst.\LANG}

\index{MIPS!\Instructions!BREAK}
\EN{Here we see here a new instruction: BREAK. It just raises an exception.}
\RU{И кстати, мы видим новую инструкцию: BREAK. Она просто генерирует исключение.}
\EN{In this case, an exception is raised if the divisor is zero (it's not possible to divide by zero 
in conventional math).}
\RU{В этом случае, исключение генерируется если делитель 0 (потому что в обычной математике нельзя
делить на ноль).}
\EN{But GCC probably did not do very well the optimization job and did not see that \$V0 is never zero.}
\RU{Но компилятор GCC наверное не очень хорошо оптимизировал код, и не заметил, что \$V0 не бывает нулем.}
\EN{So the check is left here}\RU{Так что проверка осталась здесь}.
\EN{So if \$V0 is zero somehow, BREAK is to be executed, signaling to the \ac{OS} about the exception.}
\RU{Так что если \$V0 будет каким-то образом 0, будет исполнена BREAK, сигнализирующая в \ac{OS} 
об исключении.}
\index{MIPS!\Instructions!MFLO}
\EN{Otherwise, MFLO executes, which takes the result of the division from the LO register and copies it in \$V0.}
\RU{В противном случае, исполняется MFLO, берущая результат деления из регистра LO и копирующая его в \$V0.}

\index{MIPS!\Registers!LO}
\index{MIPS!\Registers!HI}
\EN{By the way, as we may know, the MUL instruction leaves the high 32 bits of 
the result in register HI and the low 32 bits in register LO.}
\RU{Кстати, как мы уже можем знать, инструкция MUL оставляет старшую 32-битную часть результата
в регистре HI и младшую 32-битную часть в LO.}
\EN{DIV leaves the result in the LO register, and remainder in the HI register.}
\RU{DIV оставляет результат в регистре LO и остаток в HI.}

\index{MIPS!\Instructions!MFHI}
\EN{If we alter the statement to}\RU{Если изменить выражение на} \q{a \% 9}, 
\EN{the MFHI instruction is to be used here instead of MFLO.}
\RU{вместо инструкции MFLO будет использована MFHI.}

\fi

