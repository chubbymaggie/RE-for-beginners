\subsection{\RU{Инкапсуляция}\EN{Encapsulation}}

\RU{Инкапсуляция\EMDASH{}это сокрытие данных в \IT{private} секциях класса, например, чтобы разрешить доступ к ним только 
для методов этого класса, но не более.}\EN{Encapsulation is hiding the data in the \IT{private} sections of the class, 
e.g. to allow access to them only from this class methods.}

\RU{Однако, маркируется ли как-нибудь в коде тот сам факт, что некоторое поле\EMDASH{}приватное, 
а некоторое другое\EMDASH{}нет?}
\EN{However, are there any marks in code the about the fact that some field is private and
some other\EMDASH{}not?}

\RU{Нет, никак не маркируется.}\EN{No, there are no such marks.}

\RU{Попробуем простой пример:}\EN{Let's try this simple example:}

\lstinputlisting{\CURPATH/classes/classes2_0.cpp}

\RU{Снова скомпилируем в MSVC 2008 с опциями \Ox и \Obzero и посмотрим код метода \TT{box::dump()}:}
\EN{Let's compile it again in MSVC 2008 with \Ox and \Obzero options and see the \TT{box::dump()} method code:}

\lstinputlisting{\CURPATH/classes/classes2_1.asm}

\RU{Разметка полей в классе выходит такой:}\EN{Here is a memory layout of the class:}

\begin{center}
\begin{tabular}{ | l | l | }
\hline
  \tableheader{} \\
\hline
  +0x0 & int color \\
\hline
  +0x4 & int width \\
\hline
  +0x8 & int height \\
\hline
  +0xC & int depth \\
\hline
\end{tabular}
\end{center}

\RU{Все поля приватные и недоступные для модификации из других функций, но, зная эту разметку, 
сможем ли мы создать код модифицирующий эти поля?}\EN{All fields are private and not allowed to be accessed from any other
function, but knowing this layout, can we create code that modifies these fields?} 

\RU{Для этого добавим функцию \TT{hack\_oop\_encapsulation()}, которая если обладает 
приведенным ниже телом, то просто не скомпилируется:}%
\EN{To do this we'll add the \TT{hack\_oop\_encapsulation()} function, 
which is not going to compile if it looked like this:}

\lstinputlisting{\CURPATH/classes/classes2_2.cpp.\LANG}

\RU{Тем не менее, если преобразовать тип \IT{box} к типу \IT{указатель на массив int}, 
и если модифицировать полученный массив \Tint{}-ов, тогда всё получится.}
\EN{Nevertheless, if we cast the \IT{box} type to a \IT{pointer to an int array},
and we modify the array of \Tint{}-s that we have, we can succeed.}

\lstinputlisting{\CURPATH/classes/classes2_3.cpp}

\RU{Код этой функции довольно прост\EMDASH{}можно сказать, функция берет на вход указатель на массив \Tint{}-ов и 
записывает \IT{123} во второй \Tint{}:}
\EN{This function's code is very simple\EMDASH{}it can be said that the function takes a pointer to an array of \Tint{}-s for input
and writes \IT{123} to the second \Tint{}:}

\lstinputlisting{\CURPATH/classes/classes2_5.asm}

\RU{Проверим, как это работает:}\EN{Let's check how it works:}

\lstinputlisting{\CURPATH/classes/classes2_4.cpp}

\RU{Запускаем:}\EN{Let's run:}

\begin{lstlisting}
this is box. color=1, width=10, height=20, depth=30
this is box. color=1, width=123, height=20, depth=30
\end{lstlisting}

\RU{Выходит, инкапсуляция\EMDASH{}это защита полей класса только на стадии компиляции.}
\EN{We see that the encapsulation is
just protection of class fields only in the compilation stage.}
\RU{Компилятор ЯП \Cpp не позволяет сгенерировать код прямо
модифицирующий защищенные поля, тем не менее, используя \IT{грязные трюки}\EMDASH{} 
это вполне возможно.}
\EN{The \Cpp compiler is not allowing the generation of code that modifies protected 
fields straightforwardly, nevertheless,
it is possible with the help of \IT{dirty hacks}.}

