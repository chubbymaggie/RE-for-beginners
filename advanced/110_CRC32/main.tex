\chapter{\RU{Пример вычисления CRC32}\EN{CRC32 calculation example}}
\index{CRC32}
\label{sec:CRC32}

\newcommand{\URLCRCSRC}{\url{http://go.yurichev.com/17327}}

\RU{Это распространенный табличный способ вычисления хеша алгоритмом 
CRC32\footnote{Исходник взят тут: \URLCRCSRC}.}
\EN{This is a very popular table-based CRC32 hash calculation 
technique\footnote{The source code was taken from here: \URLCRCSRC}.}

\lstinputlisting{\CURPATH/CRC.c}

\index{\CLanguageElements!for}
\RU{Нас интересует функция \TT{crc()}. 
Кстати, обратите внимание на два инициализатора в выражении \TT{for()}: \TT{hash=len, i=0}. 
Стандарт \CCpp, конечно, допускает это. А в итоговом коде, вместо одной операции инициализации цикла, будет две.}
\EN{We are interesting in the \TT{crc()} function only.
By the way, pay attention to the two loop initializers in the \TT{for()} statement: \TT{hash=len, i=0}.
The \CCpp standard allows this, of course.
The emitted code will contain two operations in the loop initialization part
instead of one.}

\RU{Компилируем в MSVC с оптимизацией (\Ox). 
Для краткости, я приведу только функцию \TT{crc()}, с некоторыми комментариями.}
\EN{Let's compile it in MSVC with optimization (\Ox).
For the sake of brevity, only the \TT{crc()} function is listed here, with my comments.}

\lstinputlisting{\CURPATH/CRC_2.asm.\LANG}

\RU{Попробуем то же самое в GCC 4.4.1 с опцией \Othree:}
\EN{Let's try the same in GCC 4.4.1 with \Othree option:}

\lstinputlisting{\CURPATH/CRC_gcc_O3.\LANG}

\index{x86!\Instructions!NOP}
\index{x86!\Instructions!LEA}
\RU{GCC немного выровнял начало тела цикла по 8-байтной границе, для этого добавил 
\NOP и \TT{lea esi, [esi+0]} (что тоже \IT{холостая операция}). 
Подробнее об этом смотрите в разделе о npad~(\myref{sec:npad}).}
\EN{GCC has aligned the loop start on a 8-byte boundary by adding \NOP and \TT{lea esi, [esi+0]}
(that is an \IT{idle operation} too).
Read more about it in npad section~(\myref{sec:npad}).}

